\documentclass[12pt,spanish,makeidx]{amsbook}
\tolerance=10000
\renewcommand{\baselinestretch}{1.3}



\usepackage{t1enc}
\usepackage[spanish]{babel}
\usepackage{latexsym}
\usepackage[utf8]{inputenc}
\usepackage{verbatim}
\usepackage{multicol}
\usepackage{amsgen,amsmath,amstext,amsbsy,amsopn,amsfonts,amssymb}
\usepackage{calc}         % From LaTeX distribution
\usepackage{graphicx}     % From LaTeX distribution
\usepackage{ifthen}       % From LaTeX distribution
\usepackage{subfigure}    % From CTAN/macros/latex/contrib/supported/subfigure
\usepackage{pst-all}      % From PSTricks
\usepackage{pst-poly}     % From pstricks/contrib/pst-poly
\usepackage{multido}      % From PSTricks
\input{random.tex}        % From CTAN/macros/generic


%% \theoremstyle{plain} %% This is the default
\oddsidemargin 0.0in \evensidemargin -1.0cm \topmargin 0in
\headheight .3in \headsep .2in \footskip .2in
\setlength{\textwidth}{16cm} %ancho para apunte
\setlength{\textheight}{21cm} %largo para apunte
%\leftmargin 2.5cm
%\rightmargin 2.5cm
\topmargin 0.5 cm

\usepackage{fancyhdr}
\pagestyle{fancy}
\fancyhf{}
\fancyhead[LE,RO]{FAMAF}
\fancyhead[RE,LO]{Matemática Discreta I}
\fancyfoot[LE,RO]{\leftmark}
\fancyfoot[RE,LO]{\thepage}

\renewcommand{\headrulewidth}{0.5pt}
%\renewcommand{\footrulewidth}{0.5pt}

\newcommand{\rta}{\noindent\textit{Rta: }}


\begin{document}
	
{\bf \begin{center} Práctico 3 \\ Matemática Discreta I -- Año 2019/1 \\ FAMAF \end{center}}

{\bf \begin{center} Soluciones \end{center}}

\smallskip
\begin{enumerate}

\item Hallar el cociente y el resto de la divisi\'on de:
\begin{enumerate}
\item 
135 por 23, \rta $135= 23\times 5 +20$ $q=5, r=20$
\item
-135 por 23, \rta $-135= 23\times (-6) +3$ $q=-6, r=3$
\item
135 por -23, \rta $ -135= 23\times (-6) +3$ $q=-6, r=3$
\item
-135 por -23, \rta $135= 23\times 5 +20$ $q=5, r=20$
\item
127 por 99, \rta $127=99\times1+28$, $q=1, r=28$
\item
-98 por -73. \rta $ 98=73\times 1+25$, $q=1, r=25$
\end{enumerate}
\smallskip

\item 
\begin{enumerate}
	\item Si $a=b\cdot q+r$, con $b \le r <2 b$, hallar el cociente y el resto de la divisi\'on de $a$ por $b$.
	
	\rta $a = b \cdot (q+1) + r-b$, con $0 \le r-b < b$ por lo tanto el cociente es $q+1$ y el resto  $r-b$.
	
	
	\item Repetir el ejercicio anterior, suponiendo ahora que $-b \le r < 0$.  
	
	\rta  $a = b \cdot (q-1) + r+b$, con $0 \le r+b < b$ por lo tanto el cociente es $q-1$ y el resto  $r+b$.
\end{enumerate}


\smallskip
\item Dado $m\in \mathbb N$ hallar los restos posibles de $m^2$ y $m^3$ en la division por $3,4,5,7,8, 11$.


\rta El resto del cuadrado (cubo)  de $m$ es el resto del cuadrado (cubo)  del resto de $m$.
Por lo tanto hay que calcular los restos de $r^2$ con $0\le r\le m$.
Así tenemos para $m=3, r\in\{0, 1\}$; $m=4, r\in\{0,1\}$; $m=5, r\in\{0,1,4\}$; $m=7,r\in\{0,1,4,2\}$;
$m=8, r\in\{0,1,4\}$; $m=11, r\in\{0,1,4,9,5,3\}$.

\smallskip

\item Expresar en base 10 los siguientes enteros:
\begin{enumerate}
	\item 
	$(1503)_6$  \rta  $1 \times 6^3+ 5 \times 6^2+0 \times 6^1+3 \times 6^0=1  \times 216+ 5 \times 36 +3=399$
	\item
	$(1111)_2$  \rta  $2^3+2^2+2^1+2^0= 8+4+2+1=15$
	\item
	$(1111)_{12}$  \rta  En este ejercicio (y en otro más abajo) usaremos  que 
	\begin{equation*}
		1+ q^1 +q^2 +\cdots +q^n =  \frac{q^{n+1} -1}{q-1}\quad \text{(serie geométrica)}.
	\end{equation*}
	Por lo tanto,
	\begin{align*}
		12^3+12^2+12^1+1 &=\frac{12^4-1}{11}=\frac{144^2-1}{11}=\frac{143\times145}{11}\\
		&= \frac{143}{11} \times 145  = 13 \times 145.
	\end{align*}
	 
	\item
	$(123)_4$  \rta  $1\times 16+2\times 4+3=27$.
	\item
	$(12121)_3$  \rta $3^4+2\times 3^3+3^2+2\times 3^1+3=81+54+9+6+3=153$.
	\item
	$(1111)_5$  \rta =$(5^4-1)/4=624/4 = 156$.
\end{enumerate}

\smallskip

\item Convertir
\begin{enumerate}
	\item
	$(133)_4$ a base 8,  
	
	\rta  Debemos primero calcular cuanto vale  $(133)_4$ en base 10 (la base usual) y luego pasarlo a base 8. Ahora bien, $(133)_4 = 4^2+3\times 4+3 = 31$ y $31 = 3 \times 8 + 7$, por lo tanto  $(133)_4 =  (37)_8$.
	\item
	$(B38)_{16}$ a base 8,  
	
	\rta  Aquí usaremos que $16 = 2 \times 8$. Entonces,  $(B38)_{16} = 12\times (16)^2+3\times 16+8=6(8)^3+6\times 8+8=6\times 8^3+7\times 8=(6010)_8$.
	\item
	$(3506)_7$ a base 2, 
	
	\rta  $(3506)_7 = 3\times 7^3+5\times 7^2+6= 1280$. Ahora debemos escribir 1280 en base 2
	\begin{align*}
	1280 &= 2 \times 640 + 0 \\
	640 &= 2 \times 320 + 0 \\
	320 &= 2 \times 160 + 0 \\
	160 &= 2 \times 80 + 0 \\
	80 &= 2 \times 40 + 0 \\		
	40 &= 2 \times 20 +0\\
	20 &= 2 \times 10 + 0\\
	10&= 2 \times 5+0 \\
	5&= 2 \times 2 +1 \\
	2&= 2 \times 1+ 0\\
	1 &= 2 \times 0 + 1.
	\end{align*}
	Luego $(3506)_7 = (10100000000)_2$. 
	
	Hay una forma de hacer este ejercicio más corta: observar que $1280 = 2^7\times 10 = 2^8\times 5 = 2^8\times (2^2 +1) = 2^{10} + 2^8$. 
	\item
	$(1541)_6$ a base 4. 
	
	 \rta  $6^3+5\times 6^2+4\times 6+1=54\times 4+45\times 4+6\times 4=105\times 4=420$. Luego,
	 \begin{align*}
	 	420 &= 4 \times 105 + 0 \\
	 	105 &= 4 \times 26 +1 \\
	 	26 &= 4 \times 6 +2 \\
	 	6 &= 4 \times 1 +2 \\
	 	1 &= 4 \times 0 +1.
	 \end{align*}
	 Entonces, $(1541)_6=(12210)_4$.
\end{enumerate}


\smallskip

\item Calcular: 
\begin{enumerate} 
	\item
	$(2234)_5 + (2310)_5$  
	
	\rta La suma entre números escritos en la misma base se hace de la misma forma que la suma usual, teniendo en cuenta que en base 5 tenemos, $2+3 = 10$, $3+3 = 11$, etc.
	\begin{equation*}
		\begin{array}{r}
		(2234)_5 \\
		+(2310)_5 \\
		\hline 
		(10044)_5
		\end{array}
	\end{equation*}
	

	\item
	$(10101101)_2 + (10011)_2$  \rta  $(11000000)_2$.
\end{enumerate}
\smallskip

\item Sean $a$, $b$, $c \in {\mathbb Z}$. Demostrar las siguientes afirmaciones:
\begin{enumerate}
	\item Si $ab=1$, entonces \,$a=b=1$\, \'o \,$a=b=-1$.
	
	\rta  $a$ y $b$ no pueden ser nulos y si tienen distinto signo su producto es negativo, podemos suponer que son ambos positivos o negativos. Si $a>1$ entonces $1=ab>b>0$ es absurdo. Igualmente si $a<-1$ entonces $1=ab >-b>0$. Luego $a=\pm1$ y entonces $b=\pm1$.
	
	\item Si $a,b \neq 0$,  $a| b$\, y \,$b | a$, entonces \,$a=b$\, \'o \,$a=-b$.
	
	\rta Tenemos $b\vert a \Rightarrow a=bq$ y   $a\vert b\Rightarrow b=ap$ luego $a=apq$ y  $a\neq0 \Rightarrow pq=1$. El inciso a)  dice que $p=q=1$ o $p=q=-1$ de donde se sigue el resultado buscado.
	
	\item Si $a | 1$, entonces \,$a=1$\, \'o \,$a=-1$.
	
	\rta Este es un corolario del inciso \textit{b)} tomando $b=1$ ya que $1\vert a$, $\forall a\in \mathbb{Z}$.
	
	\item Si $a \neq 0$, $a | b$\, y \,$a | c$, entonces \,$a | (b+c)$\, y \,$a | (b-c)$.
	
	\rta Como $b=aq$ y $c=ap$, $b\pm c=a(q\pm p)$.
	
	\item Si $a \neq 0$, $a | b$\, y \,$a | (b+c)$, entonces \,$a | c$.
	
	\rta Se puede usar el inciso anterior con $b=0$.
	
	\item Si $a \neq 0$\, y \,$a | b$, entonces \,$a| b\cdot c$.
	
	\rta Si $b=aq$ entonces $b.c=aqc \Rightarrow a\vert b.c$.
\end{enumerate}

\smallskip

\item Dados $b,c$ enteros, probar las siguientes propiedades:
\begin{enumerate}
	\item  $0$\, es par y $1$\, es impar.
	
	\rta $0=2\times 0$ y $1=2\times 0+1$.
	
	\item  Si $b$ es par y \,$b \mid c$, entonces $c$ es par.  (Por lo tanto, si $b$ es par, tambi\'en lo es $-b$).
	
	\rta $b=2q, c=bp\Rightarrow c=2qp \Rightarrow c$ es par. ($b\vert -b$).
	
	\item  Si $b$ y $c$ son pares, entonces $b+c$ tambi\'en lo es. %(Por lo tanto, la suma de una cantidad cualquiera de números pares es par).
	
	\rta $2\vert b, 2\vert c \Rightarrow 2\vert b+c$.
	
	\item  Si un número par divide a 2, entonces ese número es 2\, \'o \,$-2$.
	
	\rta Dicho número $a$ no puede ser 0 y por el ejercicio 7 b) $2\vert a$ y $a\vert 2$ entonces $a= \pm2$.
	
	\item  La suma de un número par y uno impar es impar.
	
	\rta $a=2q, b=2p+1 \Rightarrow a+b=2(q+p)+1$.
	
	\item $b + c$ es par si y  sólo si $b$ y $c$ son ambos pares o ambos impares.
	
	\rta $b=2q, c=2p \Rightarrow b+c=2(q+p); b=2q+1, c=2p+1\Rightarrow b+c=2q+1+2p+1=2(q+p+1)$.
\end{enumerate}

\smallskip

\item Sea $n\in \mathbb Z$. Probar que $n$ es par si y s\'olo si $n^2$ es par.

\rta $n=2q \Rightarrow n^2=2(2q^2)$. $n=2q+1\Rightarrow n^2=4q^2+4q+1=2(2q^2+2q)+1$.

\smallskip

\item Probar que $n(n+1)$ es par para todo $n$ entero.

\rta Si $ n=2q, n(n+1)=2q(2q+1)$ es par. Si $n=2q+1, n(n+1)=(2q+1)(2q+1+1)=2(q+1)(2q+1)$.

\smallskip

\item Sean $a$, $b$, $c \in {\mathbb Z}$. ¿Cuáles de las siguientes afirmaciones son verdaderas? Justificar las respuestas.
\begin{enumerate}
	\item $a \mid b\cdot c \Rightarrow a \mid b$\, \'o \,$a \mid c$.
	
	\rta Falso, contraejemplo: $6\vert12=4\times 3$ pero 6 no divide a 4 ni divide a 3.
	
	\item $a \mid (b+c) \Rightarrow a\mid b$\, \'o \,$a \mid c$.
	
	\rta Falso, contraejemplo: $6\vert 5=1$ per 6 no divide a 5 ni a 1.
	
	\item $a \mid c$\, y \,$b \mid c \Rightarrow a\cdot b \mid c$.
	
	\rta Falso, contraejemplo: $6\vert 12$ y $4\vert 12$ pero 24 no divide a 12.
	
	\item $a \mid c$\, y \,$b \mid c$ $\Rightarrow (a +b) \mid c$.
	
	\rta  Falso, contraejemplo: $2\vert 6$ y $3\vert 6$ pero 5=2+3 no divide a 6.
	
	\item $a$, $b$, $c>0$\, y \,$a=b\cdot c$, entonces  \,$a \ge b$ y $a \ge c$.
	
	\rta  Verdadero, $b\ge1 \Rightarrow a= bc\ge c$ y $ c\ge1 \Rightarrow bc\ge b$.
\end{enumerate}


\smallskip

\item Probar que cualquiera sea $n \in {\mathbb N}$:
\begin{enumerate}
	\item $3^{2n+2}+ 2^{6n+1}$ es múltiplo de 11.
	
	\rta $3^{2n+2} + 2^{6n+1}=9^n \cdot 9+64^n \cdot 2$. Como el resto de dividir 64 por 11 es 9, tenemos que $9^n \cdot 9+64^n \cdot 2$ es divisible por 11 si $9^n \cdot 9+9^n \cdot 2$ lo es y este último es $9^n(9+2)$ que claramente es divisible por 11.
	
\textit{Rta	Alternativa}: podemos probar por inducción. Si $n=0$ es claro. Supongamos $11\vert 3^{2n+2} + 2^{6n+1}$ (HI). Debemos probar que  
$$11\vert 3^{2(n+1)+2} + 2^{6(n+1)+1} = 11\vert 3^{2n+4} + 2^{6n+7}.$$
Ahora bien,
\begin{align*}
	3^{2n+4} + 2^{6n+7} &=   3^23^{2n+2} + 2^62^{6n+1} \\
	&= 9\cdot3^{2n+2} + 64\cdot2^{6n+1} \\
	&= 	9(3^{2n+2} + 2^{6n+1}) +55\cdot 2^{6n+1} .
\end{align*}
   Es claro que el primer término es divisible por 11 por HI. El segundo término es $55\cdot 2^{6n+1}$ que es divisible por 11, pues 55 lo es. Concluyendo $9(3^{2n+2} + 2^{6n+1}) +55\cdot 2^{6n+1} $ es divisible por  11 y por lo tanto  $ 11\vert 3^{2n+4} + 2^{6n+7}$ lo es.
	
	\item $3^{2n+2} - 8n - 9$\, es divisible por 64.
	
	\rta Lo haremos por inducción. El caso base es $n=1$ y  en ese caso debemos ver que $64|3^{2\cdot 1+2} - 8\cdot 1 - 9 = 3^4-8-9 = 81-17 =64$, lo cual esta bien.
	
	Supongamos que $64|3^{2n+2} - 8n - 9$ (HI),  entonces debemos probar que $64|3^{2(n+1)+2} - 8(n+1) - 9 = 9\cdot 3^{2n+2} - 8n -17$. 
	
	Ahora bien   
	\begin{align*}
		9\cdot 3^{2n+2} - 8n -17 &=  9\cdot (3^{2n+2} - 8n - 9 + 8n + 9) - 8n -17 \\
		&=   9\cdot (3^{2n+2} - 8n - 9) + 9\cdot(8n + 9) - 8n -17 \\
		&=  9\cdot (3^{2n+2} - 8n - 9) + 72n + 81 - 8n -17 \\
		&= 9\cdot (3^{2n+2} - 8n - 9) + 64(n+1).
	\end{align*}
	El primer término es múltiplo de 64 por HI y  el segundo es $64(n+1)$ que claramente es múltiplo de 64.

\end{enumerate}

\smallskip

\item Decir si es verdadero o falso justificando:
\begin{enumerate}
	\item $3^n+1$\, es múltiplo de $n$, $\forall n \in {\mathbb N}$.
	
	\rta Falso, contraejemplo $n=3$.
	
	\item $3n^2+1$\, es múltiplo de 2, $\forall n \in {\mathbb N}$.
	
	\rta Falso, contraejemplo $n=2$.
	
	\item $(n+1)\cdot (5n+2)$\, es múltiplo de 2, $\forall n \in {\mathbb N}$.
	
	\rta Verdadero, si $n$ es par,  $5n+2$ es par y por lo tanto $(n + 1) \cdot (5n + 2)$ es múltiplo de $2$.	Si $n$ es impar $n+1$ es par y $2\vert (n + 1) \cdot (5n + 2)$.
\end{enumerate}

\smallskip

\item Probar que para todo $n \in {\mathbb Z}$, $n^2 + 2$ no es divisible por 4.

\rta Si $n$ es impar $n^2+2$ es impar y por lo tanto no es divisible por 4.
Si $n$ es par $n^2$ es divisible por 4 y como 4 no divide a 2 entonces no divide a $n^2+2$ .

\smallskip
\item Probar que todo entero impar que no es múltiplo de 3, es de la forma $6m\pm 1$, con $m$ entero.

\rta Como $n$ no es divisible por 3, debe ser $n=3q\pm1$. Si $q$ fuese impar entonces $n$ sería par, por lo tanto $q=2m$  y tenemos el resultado.

\smallskip

\item 
\begin{enumerate}
	\item Probar que el producto de tres enteros consecutivos es divisible por 6.
	
	\rta Como los pares e impares se alternan dados dos consecutivos uno de ellos debe ser par. 
	Similarmente cada tres consecutivos habrá uno que es divisible por 3 (ya que al dividirlos por 3 sus restos son tres números distintos entre 0 y 2, o sea que uno de los restos debe ser 0). Entonces $n(n+1)(n+2)$ tiene que ser divisible por 2 y por 3 y como estos son coprimos debe ser divisible por 6.
	
	\item Probar que el producto de cuatro enteros consecutivos es divisible por 24.
	
	\rta Como en el ejercicio anterior, ahora tenemos que uno de los números es divisible por 4 y otro de los restantes es divisible por 2. Entonces el producto es divisible por 8 y también hay uno que es múltiplo de 3 por lo cual el producto es divisible por $24=3\times8$.
	
	\textit{Rta Alternativa:} el producto de cuatro enteros consecutivos es de la forma $n(n-1)(n-2)(n-3)$. Ahora bien,
	\begin{equation*}
		\binom{n}{4} = \frac{n!}{(n-4)!4!} = \frac{n(n-1)(n-2)(n-3)}{4!}.
	\end{equation*}
	Por un teorema de la teórica sabemos que $\binom{n}{4}$ es un número entero, por lo tanto $ \frac{n(n-1)(n-2)(n-3)}{4!}$ es entero, lo cual quiere decir que $4! | n(n-1)(n-2)(n-3)$ (y $4! = 24$). 
	
\end{enumerate}



\smallskip
\item Probar que si $a$ y $b$ son enteros entonces $a^2+b^2$ es divisible por 7 si y s\'olo si $a$ y $b$ son divisibles por 7. ¿Es lo mismo cierto para 3? ¿Para 5?

\rta Los restos posibles de dividir por 7 son 0,1,2,3,4,5,6. Los restos de sus cuadrados son 0,1,4,2. La única suma de dos de ellos que da un múltiplo de 7 es $0+0=0$. Luego $a^2+b^2$ sólo puede ser divisible por 7 si $a$ y $b$ lo son.
En el caso de 3 tenemos tenemos que los restos de cuadrados posibles son 0 y 1 y para que la suma de 0 solo puede ser $0+0$ como en el caso anterior. Para el caso 5 tenemos $1^2+2^2$ es divisible por 5 pero 1 y 2 no lo son.

\smallskip


\item Encontrar $(7469,2464)$, $(2689,4001)$, $(2447,-3997)$, $(-1109,-4999)$.

\noindent\textit{Rta 1: }  $7469=3\cdot 2464+ 77$, $2464= 32\cdot77$ Por lo tanto $(2469,2464)=77$.

\noindent\textit{Rta 2: }  $4001=2689+ 1312$, $2689=2\cdot 1312+65$,  $1312=20\cdot 65+12$,  $65=5\cdot 12+5$,  $12=2\cdot 5+2$, $ 5=2\cdot 2+1$. Por lo tanto$ (2689, 4001)=1$.

\noindent\textit{Rta 3: } $-3997=(-2)2447+897$,  $2447=2\cdot 897+653$,  $897=653+244$,  $653=2\cdot244+165$, $244=165+79$,  $165=2\cdot79+7$, $79=11\cdot7+2$, $7=3\cdot+1$. Por lo tanto $(2447,-3997)=1$.

\noindent\textit{Rta 4: } $4999=4\cdot1109+ 563; 1109=2\cdot 563-17; 563=33\cdot 17+2; 17=8\cdot 2+1$. Por lo tanto $(-1109,-4999)=1$.
\smallskip

\item
Calcular el máximo común divisor y expresarlo como combinaci\'on lineal de los
números dados, para cada uno de  los siguientes pares de números:
\begin{enumerate}
	\item 	14 y 35, \rta $35=2\cdot14+7; 14=2\cdot7; (14, 35)=-2\cdot 14+35$.
	\item 11 y 15, \rta  $15=11+4; 11=2\cdot4+3; 4=3+1; 1= 4-3=$\par\quad\quad\quad\quad\quad$=15-11- (11-2\cdot(15-11))=3\cdot 15-4\cdot11$.
	\item 12 y 52, \rta  $(12, 52)=4=52-4\cdot12$.
	\item  12 y -52, \rta  $(12, -52)=4=-4\cdot 12+(-52)$.
	\item 12 y 532,  \rta  $ (12,532)=4=-44\cdot 12+532$.
	\item  725 y 441,
	
	\rta \begin{align*}
	725 &= 441 \cdot 1 + 284 &&\Rightarrow& 284 &= 725 - 441 \\
	441 &= 284 \cdot 1 +157 &&\Rightarrow& 157 &= 441-284 \\
	284 &= 157 \cdot 1 + 127 &&\Rightarrow& 127 &=284-157 \\
	157 &= 127 \cdot 1 + 30 &&\Rightarrow& 30 &= 157 - 127 \\
	127 &= 30 \cdot 4  + 7 &&\Rightarrow& 7 &= 127-30 \cdot 4 \\
	30 &= 7 \cdot 4 + 2  &&\Rightarrow& 2 &= 30 -7 \cdot 4 \\
	7 &= 2 \cdot 3 + 1  &&\Rightarrow& 1 &= 7- 2 \cdot 3 \\
	2 &= 1 \cdot 2 + 0.
	\end{align*} 
	Luego $(725,441)=1$ y
	\begin{align*}
	1 &= 7- 2  \cdot 3 \\
	& =  7- (30 -7 \cdot 4)  \cdot 3 = 7 \cdot 13 - 30 \cdot 3 \\
	&= (127-30 \cdot 4 ) \cdot 13 - 30 \cdot 3= 127\cdot 13 - 55 \cdot 30 \\
	&=  127\cdot 13 - 55 \cdot (157 - 127) = 68\cdot 127 - 55\cdot 157 \\
	&=  68\cdot (284-157) - 55\cdot 157 = 68\cdot 284 - 123 \cdot 157 \\
	&= 68\cdot 284 - 123 \cdot (441-284)= 191\cdot 284 - 123\cdot 441\\
	&= 191\cdot (725 - 441) - 123\cdot 441= 191 \cdot 725 - 314 \cdot 441.	
	\end{align*}
	Es decir, $1 = 191 \cdot 725 - 314 \cdot 441$.
	
	\item 606 y 108.
	
	\rta
	\begin{align*}
	606 &= 108 \cdot 5 + 66 &&\Rightarrow& 66 &= 606 -108 \cdot 5 \\
	108 &= 66 \cdot 1 +42 &&\Rightarrow& 42 &= 108 - 66 \\
	66 &= 42 \cdot 1 + 24 &&\Rightarrow& 24 &= 66 - 42 \\
	42 &= 24 \cdot 1 + 18 &&\Rightarrow& 18 &= 42 - 24 \\
	24 &= 18 \cdot 1  + 6 &&\Rightarrow& 6 &= 24 -18 \\
	18 &= 6 \cdot 3 + 0
	\end{align*} 
	Luego $(606,108) =6$ y 
	\begin{align*}
	6 &= 24 -  18 \\
	&= 24 - (42-24)  = 2 \cdot 24 - 42 \\
	& = 2 \cdot (66-42) - 42 = 2 \cdot 66 -3 \cdot 42 \\
	& = 2 \cdot 66 -3 \cdot (108 - 66) = 5\cdot 66  - 3\cdot 108 \\
	& = 5\cdot (606 -108 \cdot 5)  - 3\cdot 108  = 5\cdot 606 - 28  \cdot 108.
	\end{align*}
	Es decir, $6 =  5 \cdot 606 -  28 \cdot 108$.
\end{enumerate}



\smallskip

\item Probar que no existen enteros $x$ e $y$ que satisfagan $x+y=100$ y $(x,y)=3$.

\rta Si $(x,y)=3$ entonces $3\vert x, 3\vert y$ y por lo tanto $3\vert x+y=100$, absurdo. 

\smallskip

\item %Si $(a,b)=1$. Probar:
\begin{enumerate}
	\item Sean $a$ y $b$ coprimos. Probar que si $a\mid b\cdot c$ entonces $a \mid c$.
	
	\rta Como $a$ y $b$ son coprimos existen $r$ y $s$ tales que $1=ra+sb$ por lo tanto $c=rac+sbc$ y como $a$ divide ambos sumandos, $a\vert c$.
	
	\item Sean $a$ y $b$ coprimos. Probar que si $a \mid c$ y $b \mid c$, entonces $a\cdot b \mid c$.
	
	\rta Como $a$ y $b$ son coprimos existen $r$ y $s$ tales que $1=ra+sb$. Además $ap=c=bq$, entonces $c=rac+sbc=rabq+sbap=(rq+sp)ab$. Por lo tanto $ab\vert c$.
	
\end{enumerate}



\smallskip

\item Probar que si $n \in {\mathbb Z}$, entonces los números $2n+1$ y $\dfrac{n(n+1)}{2}$ son coprimos.

\rta Si $a$ es coprimo con $b$ y con $c$ entonces $a$ es coprimo con $bc$, ya que un primo que divida a $b\cdot c$ debe dividir a $b$ o a $c$ y entonces no puede dividir a $a$. Como $2n+1$ es coprimo con $n$ y con $n+1$ entonces es coprimo con $n(n+1)$ y por lo tanto es coprimo con $\frac{n(n+1)}{2}$.

Más explícitamente: $1=2n+1-2\cdot n; 1=2(n+1)-(2n+1) \Rightarrow (2n+1,n)=1=(2n+1, n+1)$.
Entonces  $1=(2n+1-2\cdot n)(2(n+1)-(2n+1))=(2n+1)(1+2n)-4n(n+1)=(2n+1)(1+2n)-8\frac{n(n+1)}{2}$ y esto implica $2n+1$ y $\frac{n(n+1)}{2}$ son coprimos.

\noindent\textit{Rta Alternativa: } si $p$ es un primo que divide a $\frac{n(n+1)}{2}, p$ debe dividir a $n(n+1)$ y por ser primo debe dividir a $n$ o a $n+1$. Si además se pide que $p\vert 2n+1$ que es coprimo con $n$ y $n+1$, entonces $p\vert 1$ absurdo.

\smallskip


\item Encontrar todos los enteros positivos $a$ y $b$ tales que $(a,b)=10$ y $[a,b]=100$.

\rta Es claro que $(a/10,b/10)=1$ y $[a/10,b/10] = 10$. Como $[a/10,b/10] = \displaystyle\frac{(a/10)(b/10)}{(a/10,b/10)}= (a/10)(b/10)$,  tenemos que $(a/10)(b/10) = 10$, por lo tanto $\{a/10,b/10\}\in \{\{1,10\},\{2.5\}\}$. Es decir, $a=10, b=100$ ó $a=20, b=50$ ó al revés.




\smallskip
\item
\begin{enumerate}
	\item Probar que si $d$ es divisor común de $a$ y $b$, entonces $\dfrac{(a,b)}{d} = \left(\dfrac{a}{d}, \dfrac{b}{d}\right)$.
	
	 \rta  $d\vert a, d\vert b\Rightarrow d\vert(a,b)\Rightarrow (a,b)=dq$. Como $a=(a,b)r; b=(a, b)s$ con $(r, s)=1$, se tiene $a/d=qr; b/d=qs$ con $(r, s)=1$. Por lo tanto $(a/d, b/d)=q=(a,b)/d$.
	
	\item Probar que si $a,b\in \mathbb Z$ no nulos, entonces  $\displaystyle \frac a{(a,b)}$ y $\displaystyle \frac b{(a,b)}$ son coprimos.
	
	 \rta  usar el inciso anterior con $d=(a,b)$.
\end{enumerate}

\smallskip

\item Probar que 3  y 5 son números primos.

 \rta  3 no es divisible por 2 y 5 no es divisible por 2 ni por 3.
\smallskip

\item  Dar todos los números primos positivos menores que 100.

\rta  2,3,5,7, están en la lista. Por  el criterio de la raíz debemos ver cuales números del  8 al 100 no son divisibles  por 2, 3, 5 ni 7. Estos son: 11, 13, 17, 19, 23, 29, 31, 37, 41, 43, 47, 53, 59, 61, 67, 71, 73, 79, 83, 89, 97.
 
\smallskip

\item Determinar con el criterio de la raíz cuáles de los siguientes números son primos: 113, 123, 131, 151, 199, 503.

 \rta  $\sqrt{113}< 11$ y 113 no es divisible por 2, 3, 5 y 7, por lo tanto es primo. $123=3\cdot 41$ luego no es primo. 131 es primo, pues $\sqrt{131}<12$ y 131 no es divisible por 2, 3, 7 y 11. 151 es primo, pues $\sqrt{151}<13$ y 151 no es divisible por 2, 3, 7 y 11. 199 es primo, pues $\sqrt{199}<14$ y 199 no es divisible por 2, 3, 7, 11 y 13. 503 es primo, pues $\sqrt{503} \sim 22.42...<23$ y 503 no es divisible por 2, 3, 7, 11, 13, 17 y 19.
 
\smallskip

\item Si $a\cdot b$ es un cuadrado y $a$ y $b$ son coprimos, probar que $a$ y $b$ son cuadrados.

\rta  Si un primo $p$ divide a $a$ entonces divide a $ab$ y por ser este un cuadrado $p^2\vert ab$, por ser coprimos $p$ no divide a $b$ y entonces $p^2$ debe dividir a $a$. El resultado se sigue por el principio de buen orden tomando el $ab$ más chico que contradice la proposición y considerando $ab/p^2, a/p^2, b$.


\smallskip
\item Probar que si $p_k$ es el $k$-\'esimo primo positivo entonces $$p_{k+1}\leq p_1\cdot p_2\cdot \cdots \cdot p_k+1$$

 \rta  el miembro de la derecha no es divisible por ninguno de los primeros $k$ primos luego o es el $k+1$-ésimo primo, o es divisible por un primo mayor que este. Por lo tanto debe ser un número mayor o igual que el $k+1$-ésimo primo.
 
\smallskip


\item Calcular el máximo común divisor y el mínimo común múltiplo de los siguientes pares de números usando la descomposición en números primos. 
\begin{enumerate}
	\item 
	$a = 12$ y $b = 15$.  \rta $a = 2^2 \cdot 3$, $b = 3 \cdot 5$, $(a,b)=3$, $[a,b]=2^2 \cdot 3 \cdot 5$.
	\item 
	$a = 11$ y $b = 13$.  \rta $(a,b)=1$, $[a,b]=11 \cdot 13 =143$.
	\item 
	$a = 140$ y $b = 150$.  \rta $a = 2^2 \cdot 5 \cdot 7$, $b = 2 \cdot 3 \cdot 5^2$,  $(a,b)= 2 \cdot 5 = 10$, $[a,b]=2^2  \cdot 3\cdot 5^2  \cdot 7=2100$
	\item 
	$a = 3^2\cdot5^2$  y $b = 2^2\cdot11$.  \rta  $(a,b)=1$, $[a,b]= 2^2\cdot3^2\cdot5^2\cdot11$ 
	
	\item 
	$a = 2^2\cdot3\cdot5$ y $b = 2\cdot5\cdot7$.  \rta  $(a,b)=2 \cdot 5$, $[a,b]=2^2\cdot3\cdot5\cdot7$.
\end{enumerate}
\end{enumerate}
\end{document}