\documentclass[12pt,spanish,makeidx]{amsbook}
\tolerance=10000
\renewcommand{\baselinestretch}{1.3}



\usepackage{t1enc}
\usepackage[spanish]{babel}
\usepackage{latexsym}
\usepackage[utf8]{inputenc}
\usepackage{verbatim}
\usepackage{multicol}
\usepackage{amsgen,amsmath,amstext,amsbsy,amsopn,amsfonts,amssymb}
\usepackage{calc}         % From LaTeX distribution
\usepackage{graphicx}     % From LaTeX distribution
\usepackage{ifthen}       % From LaTeX distribution
\usepackage{subfigure}    % From CTAN/macros/latex/contrib/supported/subfigure
\usepackage{pst-all}      % From PSTricks
\usepackage{pst-poly}     % From pstricks/contrib/pst-poly
\usepackage{multido}      % From PSTricks
\input{random.tex}        % From CTAN/macros/generic


%% \theoremstyle{plain} %% This is the default
\oddsidemargin 0.0in \evensidemargin -1.0cm \topmargin 0in
\headheight .3in \headsep .2in \footskip .2in
\setlength{\textwidth}{16cm} %ancho para apunte
\setlength{\textheight}{21cm} %largo para apunte
%\leftmargin 2.5cm
%\rightmargin 2.5cm
\topmargin 0.5 cm

\usepackage{fancyhdr}
\pagestyle{fancy}
\fancyhf{}
\fancyhead[LE,RO]{FAMAF}
\fancyhead[RE,LO]{Matemática Discreta I}
\fancyfoot[LE,RO]{\leftmark}
\fancyfoot[RE,LO]{\thepage}
 
\renewcommand{\headrulewidth}{0.5pt}
%\renewcommand{\footrulewidth}{0.5pt}
 



\begin{document}

{\bf \begin{center} Práctico 2 \\ Matemática Discreta I -- Año 2019/1 \\ FAMAF \end{center}}

\smallskip

\begin {enumerate}
%\item Contar las aplicaciones de $X_3=\{1,2,3\}$ en $X_4=\{1,2,3,4\}$.
%Mostrar que hay $m^3$ aplicaciones de $X_3$ en $X_m=\{1,2,\dots,m\}$, con $m \ge 1$.

%\

\item La cantidad de d\'\i gitos o cifras de un n\'umero se cuenta a partir del primer d\'\i gito distinto de cero. Por ejemplo, $0035010$ es un n\'umero de $5$ d\'\i gitos.
\begin{enumerate}
\item ¿Cu\'antos n\'umeros de 5 d\'\i gitos hay?
\item ¿Cu\'antos n\'umeros pares de 5 d\'\i gitos  hay?
\item ¿Cu\'antos n\'umeros de 5 d\'\i gitos existen con s\'olo un 3?
\item ¿Cu\'antos n\'umeros capic\'uas de 5 d\'\i gitos existen?
\item ¿Cu\'antos n\'umeros capic\'uas de a lo sumo 5 d\'\i gitos hay?
\end{enumerate}

\smallskip

\item ¿Cu\'antos n\'umeros de 6 cifras pueden formarse con los d\'\i gitos de 112200?

\smallskip

\item ¿Cu\'antos n\'umeros impares de cuatro cifras hay?

\smallskip

\item ¿Cu\'antos n\'umeros m\'ultiplos de  5 y menores que 4999 hay?

\smallskip

\item En los boletos viejos de \'omnibus, aparec\'\i a un {\em n\'umero} de 5 cifras (en este caso pod\'\i an empezar con 0), y uno ten\'\i a un {\it boleto capic\'ua} si el n\'umero lo era.
\begin{enumerate}
\item ¿Cu\'antos boletos capic\'uas hab\'\i a?
\item ¿Cu\'antos boletos hab\'\i a en los cuales no hubiera ning\'un d\'\i gito repetido?
\end{enumerate}

\smallskip

\item Las antiguas patentes de auto ten\'\i an una letra indicativa de la provincia y luego 6 d\'\i gitos. (En algunas provincias, Bs. As. y Capital, ten\'\i an 7 d\'\i gitos, pero ignoremos eso por el momento). Las nuevas patentes tienen 3 letras y luego 3 d\'\i gitos. ¿Con cu\'al de los dos criterios pueden formarse m\'as patentes?

\smallskip

\item Si uno tiene 8 CD distintos de Rock, 7 CD distintos de m\'usica cl\'asica y 5 CD distintos de cuartetos,
\begin{enumerate}
	\item ¿Cu\'antas formas distintas hay de seleccionar un CD?

	\item ¿Cu\'antas formas hay de seleccionar tres CD, uno de cada tipo?

	\item Un sonidista en una fiesta de casamientos planea poner 3 CD, uno a continuaci\'on de otro. ¿Cu\'antas formas distintas tiene de hacerlo si le han dicho que no mezcle m\'as de dos estilos?
\end{enumerate}

\smallskip

\item Mostrar que si uno arroja un dado $n$ veces y suma todos los resultados obtenidos, hay $\dfrac{6^n}{2}$ formas distintas de obtener una suma par.

\smallskip

\item ¿Cu\'antos enteros entre 1 y 10000 tienen exactamente un 7 y exactamente un 5 entre sus cifras?

\smallskip

\item ¿Cu\'antos subconjuntos de $\{0,1,2,\dots,8,9\}$ contienen al menos un impar?

\smallskip

\item El truco se juega con un mazo de 40 cartas, y se reparten 3 cartas a cada jugador. Obtener el 1 de espadas (el {\it macho}) es muy bueno. Tambi\'en lo es, por otros motivos, obtener un 7 y un 6 del mismo palo ({\it tener 33}). ¿Qu\'e es m\'as probable: obtener el macho, o tener 33?

\smallskip

\item ¿Cu\'antos comit\'es pueden formarse de un conjunto de 6 mujeres y 4 hombres, si el comit\'e debe estar compuesto por 3 mujeres y 2 hombres?

\smallskip

\item ¿De cu\'antas formas puede formarse un comit\'e de 5 personas tomadas de un grupo de 11 personas entre las cuales hay 4 profesores y 7 estudiantes, si:
\begin{enumerate}
	\item No hay restricciones en la selecci\'on?

	\item El comit\'e debe tener exactamente 2 profesores?

	\item El comit\'e debe tener al menos 3 profesores?

	\item El profesor $X$ y el estudiante $Y$ no pueden estar juntos en el comit\'e?
\end{enumerate}

\smallskip

\item Si en un torneo de f\'utbol participan $2n$ equipos, probar que el n\'umero total de opciones posibles para la primera fecha es $1\cdot 3\cdot 5 \cdot \dots \cdot (2n - 1)$. sugerencia: use un argumento por inducción. 

\smallskip

\item En una clase hay $n$ chicas y $n$ chicos. Dar el n\'umero de maneras de ubicarlos en una fila de modo que todas las chicas est\'en juntas.

\smallskip

\item ¿De cu\'antas maneras  distintas pueden sentarse 8 personas en una mesa circular?

\smallskip

\item \begin{enumerate}
	\item ¿De cu\'antas maneras distintas pueden sentarse 6 hombres y 6 mujeres en una mesa circular si nunca deben quedar dos mujeres juntas?
	\item \'Idem, pero con 10 hombres y 7 mujeres.
\end{enumerate}

\smallskip

\item 
\begin{enumerate}
	\item  ¿De cu\'antas formas distintas pueden ordenarse las letras de la palabra MATEMATICA
	\item \'Idem con las palabras ALGEBRA, GEOMETRIA.
	\item ¿De cu\'antas formas distintas pueden ordenarse las letras de la palabra MATEMATICA si se pide que las consonantes y las vocales se alternen?
\end{enumerate}

\smallskip

\item ¿Cuántas diagonales tiene un polígono regular de $n$ lados?

\smallskip

\item Dados $m$, $n$ y $k$ naturales tales que $m \le k \le n$, probar que se verifica
\[ \binom{n}{k}\binom{k}{m} = \binom{n}{m}\binom{n-m}{k-m}.\]

\smallskip

\item Probar que para todo $i$, $j$, $k \in {\mathbb N}_0$ vale
\[ \binom{i + j + k}{i}\binom{j+k}{j} = \frac{(i+j+k)!}{i!j!k!}\]

\smallskip

\item Demostrar que para todo $n \in \mathbb N$ vale:
\begin{enumerate}
  \item $\displaystyle{\binom{n}{0} + \binom{n}{1} + \cdots + \binom{n}{n} = 2^n}$.
\medskip
  \item $\displaystyle{\binom{n}{0} - \binom{n}{1} + \cdots + (-1)^n\binom{n}{n} = 0}$
  \end{enumerate}

\smallskip

\item Probar que para todo natural $n$ vale que \[\binom{2n}{2} = 2 \binom{n}{2} + n^2.\]

\smallskip



\item Con $20$ socios de un club se desea formar $5$ listas electorales (disjuntas).
Cada lista consta de $1$ Presidente, $1$ Tesorero y $2$ vocales.  ¿De cu\'antas
formas puede hacerse?

\smallskip

\item ¿De cu\'antas formas se pueden fotografiar $7$ matrimonios en una hilera,
de tal forma que cada hombre aparezca al lado de su esposa?

\smallskip

\item ¿De cu\'antas formas pueden distribuirse $14$ libros distintos entre dos
personas de manera tal que cada persona reciba al menos $3$ libros?
\end{enumerate}
\end{document}
