\documentclass[12pt,spanish,makeidx]{amsbook}
\tolerance=10000
\renewcommand{\baselinestretch}{1.3}



\usepackage{t1enc}
\usepackage[spanish]{babel}
\usepackage{latexsym}
\usepackage[utf8]{inputenc}
\usepackage{verbatim}
\usepackage{multicol}
\usepackage{amsgen,amsmath,amstext,amsbsy,amsopn,amsfonts,amssymb}
\usepackage{calc}         % From LaTeX distribution
\usepackage{graphicx}     % From LaTeX distribution
\usepackage{ifthen}       % From LaTeX distribution
\usepackage{subfigure}    % From CTAN/macros/latex/contrib/supported/subfigure
\usepackage{pst-all}      % From PSTricks
\usepackage{pst-poly}     % From pstricks/contrib/pst-poly
\usepackage{multido}      % From PSTricks
\input{random.tex}        % From CTAN/macros/generic


%% \theoremstyle{plain} %% This is the default
\oddsidemargin 0.0in \evensidemargin -1.0cm \topmargin 0in
\headheight .3in \headsep .2in \footskip .2in
\setlength{\textwidth}{16cm} %ancho para apunte
\setlength{\textheight}{21cm} %largo para apunte
%\leftmargin 2.5cm
%\rightmargin 2.5cm
\topmargin 0.5 cm

\newcommand{\nc}{\newcommand}
\nc{\RR}{\mathbb{R}} \nc{\HH}{\mathbb{H}} \nc{\CC}{\mathbb{C}} \nc{\ZZ}{\mathbb{Z}}
\nc{\FF}{\mathbb{F}} \nc{\NN}{\mathbb{N}} \nc{\QQ}{\mathbb{Q}} \nc{\PP}{\mathbb{P}}

\nc{\ben}{\begin{enumerate}} \nc{\een}{\end{enumerate}} \nc{\x}{\varphi}
\nc{\vs}{\vspace{1cm}}

\begin{document}

{\bf \begin{center} Matemática Discreta I -- Año 2019/1 \\Práctico 3 - Repaso\end{center}}


\smallskip

\begin{enumerate}




\item Sea $p$ primo positivo. Probar que $(p,(p-1)!)=1$.

\smallskip

\item Demostrar que $\forall n\in{\mathbb Z}$, $n>2$, existe $p$ primo tal que $n<p<n!$. (Ayuda: pensar qu\'e primos dividen a $n! - 1$.)



\smallskip

\item Dado un entero $a>0$ fijo, caracterizar aquellos n\'umeros que al dividirlos por $a$ tienen cociente igual al resto.

\smallskip


\item Probar que si $(a,4)=2$ y $(b,4)=2$ entonces $(a+b,4)=4$.

\smallskip
\item Probar que si $a,b$ son coprimos entonces $(a+b,a-b)=1 \text{
\'o } 2 $.



\smallskip



\item Completar y demostrar:

a) Si $a \in {\mathbb Z}$, entonces $[a,a]=\dots$

b) Si $a$, $b \in {\mathbb Z}$, $[a,b]=b$ si y s\'olo si $\ldots$

c) $(a,b)=[a,b]$ si y s\'olo si $\ldots$


\smallskip

\item Probar que si $d$ es un divisor com\'un de $a$ y $b$, entonces $\dfrac{[a,b]}{d} = \left[\dfrac{a}{d},\dfrac{b}{d}\right]$.



\smallskip

\item Probar que $(a+b,[a,b])=(a,b)$. %En particular, si dos n\'umeros son coprimos, tambi\'en lo son su suma y su producto.

\smallskip

\item Probar que si $(a,b)=1$ y $n+2$ es un n\'umero primo, entonces $(a+b, a^2 + b^2 - nab) = 1$ \'o $n+2$.



\smallskip

\item Si $a\cdot b$ es un cuadrado y $a$ y $b$ son coprimos, probar que $a$ y $b$ son cuadrados.



\medskip

\item Probar que $\sqrt 6$ es irracional.

%\medskip

%\item Probar que $2^{3n+4} + 7^{3n+1}$ es divisible por $9$, para todo $n \in {\mathbb N}$, $n$ impar.





\medskip

\item Hallar el menor m\'ultiplo de 168 que es un cuadrado.



\medskip

\item Probar que el producto de dos enteros consecutivos no nulos no es un cuadrado. (Ayuda: usar el Teorema Fundamental de la Aritm\'etica).

\medskip



\item ¿Existen enteros $m$ y $n$ tales que:

a) $m^4=27$? \qquad \qquad b) $m^2 = 12n^2$? \qquad \qquad c) $m^3 = 47n^3$?




\smallskip
\item Sean $a$ y $b$ enteros coprimos. Probar que
\begin{enumerate}
  \item $(a\cdot c, b)=(b,c)$, para todo entero $c$.
  \item $a^m$ y $b^n$ son coprimos, para todo $m,n\in \mathbb N$.
  \item $a+b$ y $a\cdot b$ son coprimos.
\end{enumerate}



\smallskip


\item ¿Cu\'al es la mayor potencia de $3$ que divide a $100!$? ¿En cu\'antos ceros termina el de\-sa\-rro\-llo decimal de $100!$?

\smallskip

\item Determinar todos los $p\in\NN$ tales que
\[ p,\, p+2,\, p+6,\, p+8,\, p+12,\, p+14 \]
sean todos primos.


\smallskip

\item  Sea $\{f_n\}_{n\in\NN}$ la sucesi\'on de Fibonacci, definida recursivamente por: $f_1=1,\, f_2=1$, $f_{n+1}=f_{n}+f_{n-1},\, n\geq 2$. Probar que:
\begin{enumerate}
\item $f_{3n}$ es par $\forall\, n\in\NN$.
\item $f_{3n+1}$ y $f_{3n+2}$ son impares $\forall\, n\in\NN$.
\item $f_{n+m}=f_mf_{n+1}+f_{m-1}f_n\; \forall n,m\in\NN,\, m\geq 2$.
\item $f_n\mid f_{nk}\;  \forall k\in\NN$.
\item $f_{n+1}f_{n-1}-f_n^2=(-1)^n\; \forall n\geq 2$.
\item $(f_{n+1},f_n)=1\; \forall\, n\in\NN$.
\end{enumerate}



\end{enumerate}

\end{document}

