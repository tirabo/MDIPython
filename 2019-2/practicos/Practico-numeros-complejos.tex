\documentclass[12pt,spanish,makeidx]{amsbook}
\tolerance=10000
\renewcommand{\baselinestretch}{1.3}



\usepackage{t1enc}
\usepackage[spanish]{babel}
\usepackage{latexsym}
\usepackage[utf8]{inputenc}
\usepackage{verbatim}
\usepackage{multicol}
\usepackage{amsgen,amsmath,amstext,amsbsy,amsopn,amsfonts,amssymb}
\usepackage{calc}         % From LaTeX distribution
\usepackage{graphicx}     % From LaTeX distribution
\usepackage{ifthen}       % From LaTeX distribution
\usepackage{subfigure}    % From CTAN/macros/latex/contrib/supported/subfigure
\usepackage{pst-all}      % From PSTricks
\usepackage{pst-poly}     % From pstricks/contrib/pst-poly
\usepackage{multido}      % From PSTricks
\input{random.tex}        % From CTAN/macros/generic


%% \theoremstyle{plain} %% This is the default
\oddsidemargin 0.0in \evensidemargin -1.0cm \topmargin 0in
\headheight .3in \headsep .2in \footskip .2in
\setlength{\textwidth}{16cm} %ancho para apunte
\setlength{\textheight}{21cm} %largo para apunte
%\leftmargin 2.5cm
%\rightmargin 2.5cm
\topmargin 0.5 cm

\newcommand{\nc}{\newcommand}
\nc{\RR}{\mathbb{R}} \nc{\HH}{\mathbb{H}} \nc{\CC}{\mathbb{C}} \nc{\ZZ}{\mathbb{Z}}
\nc{\FF}{\mathbb{F}} \nc{\NN}{\mathbb{N}} \nc{\QQ}{\mathbb{Q}} \nc{\PP}{\mathbb{P}}
\nc{\C}{\mathbb{C}}
\nc{\ben}{\begin{enumerate}} \nc{\een}{\end{enumerate}} \nc{\x}{\varphi}
\nc{\vs}{\vspace{1cm}}

\usepackage{fancyhdr}
\pagestyle{fancy}
\fancyhf{}
\fancyhead[LE,RO]{FAMAF}
\fancyhead[RE,LO]{Matemática Discreta I}
\fancyfoot[LE,RO]{\leftmark}
\fancyfoot[RE,LO]{\thepage}

\renewcommand{\headrulewidth}{0.5pt}
%\renewcommand{\footrulewidth}{0.5pt}




\begin{document}
	
	{\bf \begin{center} Práctico Números Complejos \\ Matemática Discreta I -- Año 2019/1 \\ FAMAF \end{center}}
	
	\smallskip


\begin{enumerate}

\item
Simplificar las siguientes expresiones:
$$\begin{array}{ll}
 \text{a) } \ \left(\dfrac{-3}{\frac{4}{5}+1}\right)^{-1}\cdot\left(\dfrac{4}{5}-1\right) + \dfrac{1}{3}, \quad &
\text{ b)} \ \dfrac{a}{2\pi-6}(\pi-3)^2 -\dfrac{2a(\pi^2-9)}{\pi-3}.
\end{array}$$

\vspace{.5cm}


\item Demostrar que  dados $z$, $z_1$, $z_2$ en $\C$ se cumple:
\[ |\bar z|= |z|, \qquad |z_1 \, z_2|= |z_1| \, |z_2|. \]

\vspace{.5cm}


\item Sean $z=1+i$ y $w=\sqrt{2}-i$. Calcular:
 \begin{enumerate}
  \item $z^{-1}$; $1/w$; $z/w$; $w/z$.

  \item $1+z+z^2+z^3+\dots+z^{2019}$.

  \item $(z(z+w)^2-iz)/w$.
 \end{enumerate}


\vspace{.5cm}


\item Sumar y multiplicar los siguientes pares de números complejos
	\begin{enumerate}
		\item $2+ 3i$ y $4$.
		\item $2+ 3i$ y $4i$.
		\item $1 + i$ y $ 1 -i$.
		\item $3-2i$ y $1 +i$. 
	\end{enumerate}

\vspace{.5cm}


 \item Expresar los siguientes n{\'u}meros complejos en la forma $a +i b$.
 Hallar el m{\'o}dulo, argumento y conjugado de cada uno de ellos y graficarlos.

 $$\begin{array}{lll}
 \text{a) }\ 2e^{\mathrm{i}\pi}-i,  \quad & \text{ b)} \  i^3 - 2i^{-7} -1, \quad &\text{ c)}\ (-2+i) (1+2i).
  \end{array}$$

\vspace{.5cm}

\item Sean $a,b\in\mathbb{C}$. Decidir si existe $z \in \mathbb{C}$ tal que:
\begin{enumerate}
  \item $z^2=b$. ?`Es \'unico? ?`Para qu\'e valores de $b$ resulta $z$ ser un n\'umero real?
  \item $z$ es imaginario puro y $z^2=4$.
  \item $z$ es imaginario puro y $z^2=-4$.
\end{enumerate}


\end{enumerate}

\end{document}



