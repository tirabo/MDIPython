\documentclass[12pt,spanish,makeidx]{amsbook}
\tolerance=10000
\renewcommand{\baselinestretch}{1.3}



\usepackage{t1enc}
\usepackage[spanish]{babel}
\usepackage{latexsym}
\usepackage[utf8]{inputenc}
\usepackage{verbatim}
\usepackage{multicol}
\usepackage{amsgen,amsmath,amstext,amsbsy,amsopn,amsfonts,amssymb}
\usepackage{calc}         % From LaTeX distribution
\usepackage{graphicx}     % From LaTeX distribution
\usepackage{ifthen}       % From LaTeX distribution
\usepackage{subfigure}    % From CTAN/macros/latex/contrib/supported/subfigure
\usepackage{pst-all}      % From PSTricks
\usepackage{pst-poly}     % From pstricks/contrib/pst-poly
\usepackage{multido}      % From PSTricks
\input{random.tex}        % From CTAN/macros/generic


%% \theoremstyle{plain} %% This is the default
\oddsidemargin 0.0in \evensidemargin -1.0cm \topmargin 0in
\headheight .3in \headsep .2in \footskip .2in
\setlength{\textwidth}{16cm} %ancho para apunte
\setlength{\textheight}{21cm} %largo para apunte
%\leftmargin 2.5cm
%\rightmargin 2.5cm
\topmargin 0.5 cm

\newcommand{\nc}{\newcommand}
\nc{\RR}{\mathbb{R}} \nc{\HH}{\mathbb{H}} \nc{\CC}{\mathbb{C}} \nc{\ZZ}{\mathbb{Z}}
\nc{\FF}{\mathbb{F}} \nc{\NN}{\mathbb{N}} \nc{\QQ}{\mathbb{Q}} \nc{\PP}{\mathbb{P}}

\nc{\ben}{\begin{enumerate}} \nc{\een}{\end{enumerate}} \nc{\x}{\varphi}
\nc{\vs}{\vspace{1cm}}

\usepackage{fancyhdr}
\pagestyle{fancy}
\fancyhf{}
\fancyhead[LE,RO]{FAMAF}
\fancyhead[RE,LO]{Matemática Discreta I}
\fancyfoot[LE,RO]{\leftmark}
\fancyfoot[RE,LO]{\thepage}
 
\renewcommand{\headrulewidth}{0.5pt}
%\renewcommand{\footrulewidth}{0.5pt}
 



\begin{document}

{\bf \begin{center} Práctico 4 \\ Matemática Discreta I -- Año 2019/1 \\ FAMAF \end{center}}

\smallskip


\begin{enumerate}
\item  
\begin{enumerate}
	\item Calcular el resto de la divisi\'on de 1599 por 39 sin tener que hacer la divisi\'on. \\(Ayuda: $1599=1600-1=40^2-1$).
	\item Lo mismo con el resto de 914 al dividirlo por 31.
\end{enumerate}


\smallskip
\item Sea $n\in\NN$. Probar que todo n\'umero de la forma $4^n-1$ es divisible por 3.

\smallskip
\item Probar que el resto de dividir $n^2$ por 4 es igual a 0 si $n$ es par y 1 si $n$ es impar.

%\smallskip

%\item Probar que si las longitudes de los lados de un tri\'angulo rect\'angulo son n\'umeros enteros, entonces los catetos no pueden ser ambos impares.

\smallskip
\item
\begin{enumerate}
\item
Probar las reglas de divisibilidad por 2, 3, 4, 5, 8, 9 y 11.% que no hayan sido probadas en el te\'orico.
\item Decir por cu\'ales de los n\'umeros del 2 al 11 son divisibles los siguientes n\'umeros:
$$ \qquad 12342  \, \qquad   \qquad  5176 \, \qquad \qquad  314573\,  \qquad  \qquad  899.$$
\end{enumerate}

%\smallskip \item Hallar los restos posibles en la divisi\'on de $n^2$ por 3.

\smallskip
\item Sean $a$, $b$, $c$ n\'umeros enteros, ninguno divisible por 3. Probar que 
$$a^2 + b^2 + c^2\equiv 0 \pmod 3.$$% es divisible por 3.


\smallskip
\item Hallar la cifra de las unidades y la de las decenas del n\'umero $7^{15}$.


\smallskip
\item Hallar el resto en la divisi\'on de $x$ por 5 y por 7 para:
 \begin{enumerate}
\item $x=1^8 + 2^8 + 3^8 + 4^8 + 5^8 + 6^8 + 7^8 + 8^8$;
\item $x=3\cdot 11\cdot 17\cdot 71\cdot 101$.
\end{enumerate}
%\end{multicols}

\smallskip
\item Hallar todos los $x$ que satisfacen:
\begin{multicols}{3}
 \begin{enumerate}
  \item $x^2 \equiv 1\,\, (4)$
\item  $x^2 \equiv x\,\, (12)$
\item  $x^2 \equiv 2 \,\, (3)$
\item  $x^2 \equiv 0\,\, (12$)
\item  $x^4 \equiv 1 \,\, (16)$
\item  $3x \equiv 1\,\, (5)$
%\item $2x\equiv 5\,\, (6)$
%\item  $3x^3 \equiv 20\,\, (8)$.
 \end{enumerate}
\end{multicols}

%\smallskip
\item Sean $a$, $b$, $m \in {\ZZ}$, $d>0$ tales que  \,$d\mid a$,\,  \,$d\mid b$\, y \,$d\mid m$. Probar que la ecuaci\'on $a\cdot x \equiv b\,( m)$ tiene
soluci\'on si y s\'olo si la ecuaci\'on
\[ \frac{a}{d}\cdot x \equiv \frac{b}{d}\,\left(\frac{m}{d}\right)\]
tiene soluci\'on.

\smallskip
\item Resolver las siguientes ecuaciones:
\begin{multicols}{3}
 \begin{enumerate}
  \item $2x \equiv -21 \,\,\, (8)$
\item $2x \equiv -12 \,\,\, (7)$
\item $3x \equiv 5\,\,\, (4)$.
 \end{enumerate}
\end{multicols}

\smallskip
\item Resolver la ecuaci\'on $221 x \equiv 85\,\, (340)$. Hallar todas las soluciones $x$ tales que $0 \le x < 340$.

\smallskip
\item 
\begin{enumerate}
	\item[(i)] Encontrar todas las soluciones de la ecuaci\'on en congruencia
	$$36\,x\equiv 8 \quad (20)$$
	usando el método visto en clase.
	\item[(ii)] Dar todas las soluciones $x$ de la ecuaci\'on anterior tales que $-8 < x < 30$.
\end{enumerate}


\smallskip
\item 
\begin{enumerate}
\item[(i)] Encontrar todas las soluciones de la ecuaci\'on en congruencia
$$21\,x\equiv 6 \quad (30)$$
usando el método visto en clase.
\item[(ii)] Dar todas las soluciones $x$ de la ecuaci\'on anterior tales que $0 < x < 35$.
\end{enumerate}





\smallskip
\item Dado $t \in {\ZZ}$, decimos que $t$ es {\it inversible m\'odulo $m$} si existe $h \in {\ZZ}$ tal que $th\equiv 1\,(\ m)$.
  \begin{enumerate}
  \item ¿Es 5 inversible m\'odulo 17?
%  \item ¿Existe alg\'un $m$ tal que $m$ sea inversible m\'odulo $m$?
  \item Probar que $t$ es inversible m\'odulo $m$, si y s\'olo si $(t,m)=1$.
  \item Determinar los inversibles m\'odulo $m$, para $m=11,12,16$.
  \end{enumerate}



%\begin{enumerate}
%\item Dar la tabla de la suma y del producto en $\ZZ_2$, $\ZZ_3$ y $\ZZ_4$.
%  \item Probar que $\ZZ_m$ es un anillo.
% \end{enumerate}

\smallskip
\item Encontrar los enteros cuyos cuadrados divididos por 19 dan resto 9.

\smallskip

\item Probar que todo n\'umero impar $a$ satisface: $a^4 \equiv 1(16)$, $a^8 \equiv 1(32)$, $a^{16}\equiv 1(64)$.\\ ¿Se puede asegurar que $a^{2^n} \equiv 1 (2^{n+2})$?

\smallskip

\item Encontrar el resto en la divisi\'on de $a$ por $b$ en los siguientes casos:
\begin{multicols}{2}
 \begin{enumerate}
\item $a=11^{13}\cdot 13^8$; \quad $b=12$;
\item $a=4^{1000}$;\quad $b=7$;
\item $a=123^{456}$; \quad  $b=31$;
\item $a=7^{83}$;\quad  $b=10$.
\end{enumerate}
\end{multicols}

%\smallskip
\item Obtener el resto en la divisi\'on de $2^{21}$ por 13; de $3^8$ por 5 y de
 $8^{25}$ por 127.
%$$ \text{ (a)  $2^{21}$\, por \,13;\qquad
 %$3^8$\, por \,5 $8^{25}$\, por \,127.}$$


%\smallskip

%\item Hallar todos los enteros que satisfacen simult\'aneamente:

%$x \equiv 1\ ( 3); $ \qquad $x \equiv 1 \ ( 5)$; \qquad $x \equiv 1\ ( 7)$.

%\smallskip
%\item Hallar el menor entero positivo que satisface simult\'aneamente las siguientes congruencias:

%$x\equiv 2\ ( 3)$; \qquad $x \equiv 3\ ( 5)$; \qquad $x \equiv 5\ (2)$.

%\smallskip

%\item Hallar 4 enteros consecutivos divisibles por \,5, 7, 9 y 11 respectivamente.

\smallskip
\item \begin{enumerate}
\item Probar que no existen enteros no nulos tales que $x^2 + y^2 = 3z^2$.
\item Probar que no existen n\'umeros racionales no nulos $a$, $b$, $r$ tales que $3(a^2 + b^2) = 7r^2$.
\end{enumerate}

\smallskip

%\item Cinco hombres recogieron en una isla un cierto n\'umero de cocos y resolvieron repartirlos al d\'\i a siguiente. Durante la noche uno de ellos decidi\'o separar su parte y para ello %dividi\'o el total en cinco partes y di\'o un coco que sobraba a un mono y se fue a dormir. Enseguida otro de los hombres hizo lo mismo, dividiendo lo que hab\'\i a quedado por cinco, dando %un coco que sobraba a un mono y retirando su parte, se fue a dormir. Uno tras otro los tres restantes hicieron lo mismo, d\'andole a un mono el coco que sobraba. A la ma\~nana siguiente %repartieron los cocos restantes, d\'andole a un mono el coco sobrante.
%>Cu\'al es el n\'umero m\'\i nimo de cocos que se recogieron?

%\smallskip

%\item La producci\'on diaria de huevos en una granja es inferior a 75.
%Cierto d\'\i a el recolector inform\'o que la cantidad de huevos recogida era tal que contando de a 3 sobraban 2, contando de a 5 sobraban 4 y
%contando de a 7 sobraban 5. El capataz, dijo que eso era imposible. >Qui\'en ten\'\i a raz\'on? Justificar.

%\smallskip

\item Probar que si \,$(a,1001)=1$\, entonces \,$1001$\, divide a \,$a^{720}-1$.


\smallskip


\


\noindent (*): ejercicios opcionales de mayor dificultad.

\smallskip

\item (*) ¿Para qu\'e valores de \,$n$\, es \,$10^n-1$\, divisible por \,$11$?


\smallskip

\item (*) Probar que para ning\'un $n\in\NN$ se puede partir el conjunto $\{n,n+1,\ldots, n+5\}$ en dos partes
disjuntas no vac\'\i as tales que los productos de los elementos que las integran sean iguales.

\smallskip

\item (*) El n\'umero \,$2^{29}$\, tiene nueve cifras y todas distintas.
¿Cu\'al d\'\i gito falta? (No est\'a permitido el uso de calculadora).



\end{enumerate}






\end{document}

\

\noindent {\bf Ejercicios de parciales y ex\'amenes anteriores:}

\

\item Decidir si las siguientes afirmaciones son verdaderas o
falsas. Justificar.

\begin{enumerate}
\item La suma de cuatro cuadrados siempre es m\'ultiplo de cuatro.

\item $7^{50}\equiv 10 \mod(13)$.

\item Existe un n\'umero entero $x$ tal que $1001 x\equiv 104 \mod(39)$.

\item Existe $n\in\NN$ tal que $4n+3$ es suma de dos cuadrados.

\item $777^{151}\equiv 7 \mod(11)$.

\item Existe un n\'umero entero $x$ tal que $1001\, x\equiv 131 \mod(39)$.

\end{enumerate}

\

\item Sean $a,b\in\ZZ$.  Probar que si $11$ divide a $a^2+5b^2$ entonces $a$ y $b$ son
tambi\'en divisibles por $11$.

\

\item  Hallar tres n\'umeros naturales consecutivos mayores que $2000$ y divisibles por
$2$, $7$ y $13$, respectivamente.

\

\item Hallar el resto de la divisi\'on por $11$ de $(61162)^{53}$.
\end{enumerate}

\end{document}
