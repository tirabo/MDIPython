\documentclass[12pt,spanish,makeidx]{amsbook}
\tolerance=10000
\renewcommand{\baselinestretch}{1.3}



\usepackage{t1enc}
\usepackage[spanish]{babel}
\usepackage{latexsym}
\usepackage[utf8]{inputenc}
\usepackage{verbatim}
\usepackage{multicol}
\usepackage{amsgen,amsmath,amstext,amsbsy,amsopn,amsfonts,amssymb}
\usepackage{calc}         % From LaTeX distribution
\usepackage{graphicx}     % From LaTeX distribution
\usepackage{ifthen}       % From LaTeX distribution
\usepackage{subfigure}    % From CTAN/macros/latex/contrib/supported/subfigure
\usepackage{pst-all}      % From PSTricks
\usepackage{pst-poly}     % From pstricks/contrib/pst-poly
\usepackage{multido}      % From PSTricks
\input{random.tex}        % From CTAN/macros/generic


%% \theoremstyle{plain} %% This is the default
\oddsidemargin 0.0in \evensidemargin -1.0cm \topmargin 0in
\headheight .3in \headsep .2in \footskip .2in
\setlength{\textwidth}{16cm} %ancho para apunte
\setlength{\textheight}{21cm} %largo para apunte
%\leftmargin 2.5cm
%\rightmargin 2.5cm
\topmargin 0.5 cm

\usepackage{fancyhdr}
\pagestyle{fancy}
\fancyhf{}
\fancyhead[LE,RO]{FAMAF}
\fancyhead[RE,LO]{Matemática Discreta I}
\fancyfoot[LE,RO]{\leftmark}
\fancyfoot[RE,LO]{\thepage}
 
\renewcommand{\headrulewidth}{0.5pt}
%\renewcommand{\footrulewidth}{0.5pt}
 



\begin{document}

{\bf \begin{center} Práctico 3 \\ Matemática Discreta I -- Año 2019/1 \\ FAMAF \end{center}}

\smallskip


\begin{enumerate}


\smallskip

\item Hallar el cociente y el resto de la divisi\'on de:

\begin{multicols}{3}
\begin{enumerate}
  \item $135$ por $23$,
	\item $-135$ por $23$,
	\item $153$ por $27$,
  \item $-135$ por $27$,
	\item $127$ por $99$,
	\item $-98$ por $73$.
\end{enumerate}
\end{multicols}

\smallskip

\item 
\begin{enumerate}
  \item Si $a=b\cdot q+r$, con $b \le r <2 b$, hallar el cociente y el resto de la divisi\'on de $a$ por $b$.
  \item Repetir el ejercicio anterior, suponiendo ahora que $-b \le r < 0$.
\end{enumerate}


\smallskip
\item Dado $m\in \mathbb N$ hallar los restos posibles de $m^2$ y $m^3$ en la division por $3,4,5,7,8, 11$.

\smallskip
\item Expresar en base 10 los siguientes enteros:
\begin{multicols}{3}
\begin{enumerate}
	\item $(1503)_6$ 
	\item $(1111)_2$ 
	\item $(1111)_{12}$
	\item $(123)_4$ 
	\item $(12121)_3$
	\item $(1111)_5$
\end{enumerate}
\end{multicols}

\smallskip

\item Convertir
\begin{multicols}{2}
\begin{enumerate}
	\item  $(133)_4$ a base 8,
	\item  $(B38)_{16}$ a base 8,
	\item  $(3506)_7$ a base 2,
	\item  $(1541)_6$ a base 4.
\end{enumerate}
\end{multicols}

\smallskip

\item Calcular: a) $(2234)_5+(2310)_5$ \qquad \qquad b)$(10101101)_2+(10011)_2$.

\smallskip

\item Sean $a$, $b$, $c \in {\mathbb Z}$. Demostrar las siguientes afirmaciones:
  \begin{enumerate}
%  \item $\forall a$,\,\, $a \mid 0$. (En particular, $0 \mid 0$).
%  \item $\forall a \neq 0$,\, $0 \not|\  a$.
  \item Si $ab=1$, entonces \,$a=b=1$\, \'o \,$a=b=-1$.
  \item Si $a,b \neq 0$,  $a| b$\, y \,$b | a$, entonces \,$a=b$\, \'o \,$a=-b$.
  \item Si $a | 1$, entonces \,$a=1$\, \'o \,$a=-1$.
  \item Si $a \neq 0$, $a | b$\, y \,$a | c$, entonces \,$a | (b+c)$\, y \,$a | (b-c)$.
  \item Si $a \neq 0$, $a | b$\, y \,$a | (b+c)$, entonces \,$a | c$.
  \item Si $a \neq 0$\, y \,$a | b$, entonces \,$a| b\cdot c$.
  \end{enumerate}

\smallskip

\item Dados $b,c$ enteros, probar las siguientes propiedades:
  \begin{enumerate}
  \item  $0$\, es par y $1$\, es impar.
  \item  Si $b$ es par y \,$b \mid c$, entonces $c$ es par.  ( Por lo tanto, si $b$ es par, tambi\'en lo es $-b$).
  \item  Si $b$ y $c$ son pares, entonces $b+c$ tambi\'en lo es. %(Por lo tanto, la suma de una cantidad cualquiera de n\'umeros pares es par).
  \item  Si un n\'umero par divide a 2, entonces ese n\'umero es 2\, \'o \,$-2$.
  \item  La suma de un n\'umero par y uno impar es impar.
  \item $b + c$ es par si y  sólo si $b$ y $c$ son ambos pares o ambos impares.
\end{enumerate}

\smallskip

\item Sea $n\in \mathbb Z$. Probar que $n$ es par si y s\'olo si $n^2$ es par.

\smallskip

\item Probar que $n(n+1)$ es par para todo $n$ entero.

\smallskip

\item Sean $a$, $b$, $c \in {\mathbb Z}$. ¿Cu\'ales de las siguientes afirmaciones son verdaderas? Justificar las respuestas.
  \begin{enumerate}
  \item $a \mid b\cdot c \Rightarrow a \mid b$\, \'o \,$a \mid c$.
  \item $a \mid (b+c) \Rightarrow a\mid b$\, \'o \,$a \mid c$.
  \item $a \mid c$\, y \,$b \mid c \Rightarrow a\cdot b \mid c$.
  \item $a \mid c$\, y \,$b \mid c$ $\Rightarrow (a +b) \mid c$.
  \item $a$, $b$, $c>0$\, y \,$a=b\cdot c$, entonces\, $a \ge b$ y $a \ge c$.
  \end{enumerate}


\smallskip

\item Probar que cualquiera sea $n \in {\mathbb N}$:
  \begin{enumerate}
  \item $3^{2n+2}+ 2^{6n+1}$ es m\'ultiplo de 11.
  \item $3^{2n+2} - 8n - 9$\, es divisible por 64.
  \end{enumerate}

\smallskip

\item Decir si es verdadero o falso justificando:
  \begin{enumerate}
  \item $3^n+1$\, es m\'ultiplo de $n$, $\forall n \in {\mathbb N}$.
  \item $3n^2+1$\, es m\'ultiplo de 2, $\forall n \in {\mathbb N}$.
  \item $(n+1)\cdot (5n+2)$\, es m\'ultiplo de 2, $\forall n \in {\mathbb N}$.
  \end{enumerate}

\smallskip

\item Probar que para todo $n \in {\mathbb Z}$, $n^2 + 2$ no es divisible por 4.


\smallskip
\item Probar que todo entero impar que no es m\'ultiplo de 3, es de la forma $6m\pm 1$, con $m$ entero.

\smallskip

\item 
\begin{enumerate}
 \item Probar que el producto de tres enteros consecutivos es divisible por 6.
 \item Probar que el producto de cuatro enteros consecutivos es divisible por 24.
\end{enumerate}

\smallskip

\item Si $a\cdot b$ es un cuadrado y $a$ y $b$ son coprimos, probar que $a$ y $b$ son cuadrados.

\smallskip

\smallskip
\item Probar que si $a$ y $b$ son enteros entonces $a^2+b^2$ es divisible por 7 si y s\'olo si $a$ y $b$ son divisibles por 7.
¿Es lo mismo cierto para 3? ¿Para 5?

\smallskip


\item Encontrar $(7469,2464)$, $(2689,4001)$, $(2447,-3997)$, $(-1109,-4999)$.

\smallskip

\item
Calcular el m\'aximo com\'un divisor y expresarlo como combinaci\'on lineal de los
n\'umeros dados, para cada uno de  los siguientes pares de n\'umeros:
\begin{multicols}{3}
\begin{enumerate}
  \item  14 y 35, 
	\item 11 y 15, 
	\item 12 y 52,
  \item 12 y $-52$  
	\item 12 y 532.
\end{enumerate}
\end{multicols}

\smallskip

\item Mostrar que 725 y 441 son coprimos y encontrar enteros $m,n$ tales que $ m\cdot 725+ n\cdot 441= 1$.


\item Dado un entero $a$, $a\neq 0$, hallar $(0,a)$.

\smallskip

 \item Calcular el m\'aximo com\'un divisor entre 606 y 108 y
 expresarlo como combinaci\'on lineal de esos n\'umeros.



\smallskip

\item Probar que no existen enteros $x$ e $y$ que satisfagan $x+y=100$ y $(x,y)=3$.

\smallskip

\item %Si $(a,b)=1$. Probar:
\begin{enumerate}
 \item Sean $a$ y $b$ coprimos. Probar que si $a\mid b\cdot c$ entonces $a \mid c$.
 \item Sean $a$ y $b$ coprimos. Probar que si $a \mid c$ y $b \mid c$, entonces $a\cdot b \mid c$.

\end{enumerate}



\smallskip

\item Probar que si $n \in {\mathbb Z}$, entonces los n\'umeros $2n+1$ y $\dfrac{n(n+1)}{2}$ son coprimos.

\smallskip


\item Calcular el m\'\i nimo com\'un m\'ultiplo de los siguientes pares de n\'umeros% dados en el Ejercicio 1.
\begin{multicols}{3}
\begin{enumerate}
  \item $a=12$ y $b=15$. 
	\item$a=11$ y $b=13$.
	\item $a=140$ y $b=150$.
  \item $a=3^2 \cdot 5^2$ y $b=2^2 \cdot 11$.
	\item $a=2^2 \cdot 3\cdot 5$ y $b=2\cdot 5\cdot 7$.
\end{enumerate}
\end{multicols}
\smallskip


\item Encontrar todos los enteros positivos $a$ y $b$ tales que $(a,b)=10$ y $[a,b]=100$.


\smallskip
\item
\begin{enumerate}
\item Probar que si $d$ es divisor común de $a$ y $b$, entonces $\dfrac{(a,b)}{d} = \left(\dfrac{a}{d}, \dfrac{b}{d}\right)$.
\item Probar que si $a,b\in \mathbb Z$ no nulos, entonces  $\displaystyle \frac a{(a,b)}$ y $\displaystyle \frac b{(a,b)}$ son coprimos.
\end{enumerate}

\smallskip

\item Probar que 3  y 5 son n\'umeros primos.

\smallskip

\item Determinar cu\'ales de los siguientes n\'umeros son primos: 113, 123, 131, 151, 199, 503.

\smallskip

\item  Dar todos los n\'umeros primos positivos menores que 100.

\smallskip
\item Probar que si $p_k$ es el $k$-\'esimo primo positivo entonces
$$p_{k+1}\leq p_1\cdot p_2\cdot \cdots \cdot p_k+1$$



\end{enumerate}


\end{document}

