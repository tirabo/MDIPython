



\chapter[Conteo]{Conteo}

Intutivamente, diremos que un conjunto $A$ es finito si podemos contar la cantidad de elementos que tiene. En ese caso denotaremos $|A|$ la cantidad de elementos de $A$ y la llamaremos el {\em cardinal de $A$}\index{cardinal de un conjunto}.  

\begin{section}{Principios básicos}


\noindent\textbf{El principio de adición}

Se puede realizar una acción $X$ de $n$ formas distintas o, alternativamente, se puede realizar otra acción $Y$ de $m$ formas distintas. Entonces el número de formas de realizar la acción ``$X$ o $Y$'' es $n + m$.

\begin{ejemplo}\label{cine} Supongamos que una persona va a salir a pasear  y puede ir al cine donde hay $3$ películas en cartel o al teatro donde hay $4$ obras posibles. Entonces, tendrá un total de $3+4=7$ formas distintas de elegir el paseo. 
\end{ejemplo}


Este principio es el más básico del conteo y más formalmente dice que si $A$ y $B$ son conjuntos finitos disjuntos, entonces 
\begin{equation}\label{padd}
|A \cup B| =|A|+|B|.
\end{equation}
El principio es fácilmente generalizable a varios conjuntos.

\begin{proposicion}\label{principiodeadicion}
Sean $A_1,\ldots,A_n$ conjuntos finitos tal que $A_i \cap A_j = \emptyset$ cuando $i\not=j$, entonces 
\begin{equation*}
|A_1 \cup \cdots \cup A_n| =|A_1|+\cdots+|A_n|.
\end{equation*}
\end{proposicion}



\begin{proof} 
La  prueba se hace por inducción en $n$. Debemos probar 
\begin{align*}
P(n) =\; &\text{Si $A_1,\ldots,A_n$ conjuntos finitos disjuntos dos a dos, entonces }\\  &|A_1 \cup \cdots \cup A_n| =|A_1|+\cdots+|A_n|.
\end{align*}

\noindent({\em Caso base $n=1$}) En este caso no hay nada que probar pues  $|A_1|=|A_1|$.

\noindent({\em Paso inductivo}) La hipótesis inductiva es $P(k)$ y debemos probar que $P(k) \Rightarrow P(k+1)$. Si denotamos $B = A_1 \cup \cdots \cup A_k$, entonces 
$$
A_1 \cup \cdots \cup A_{k+1} = B \cup A_{k+1}
$$
Ahora bien, si $x \in B \cap A_{k+1}$, entonces $x \in A_i$ para algún $i < k+1$ y $x \in A_{k+1}$. Como $A_{i} \cap A_{k+1} = \emptyset$, se produce un absurdo  que viene de suponer que existía un elemento en $B \cap A_{k+1}$. Luego   $B \cap A_{k+1}= \emptyset$ y por el principio de adición  $|B \cup A_{k+1}| = |B|+|A_{k+1}|$. 

Por la hipótesis inductiva tenemos que 
$$
|B| = |A_1 \cup \cdots \cup A_k| =|A_1|+\cdots+|A_k|,
$$
Luego
$$
|A_1 \cup \cdots \cup A_k \cup  A_{k+1}| = |B|+|A_{k+1}| = |A_1|+\cdots+|A_k|+|A_{k+1}|.
$$
\end{proof}



Si $A$ y $B$ no son disjuntos, cuando sumamos $|A|$ y $|B|$ estamos contando $A \cap B$ dos veces. Entonces, para
obtener la respuesta correcta debemos restar $|A \cap B|$ y obtenemos
$$
|A \cup B| = |A|+|B| - |A \cap B|.
$$
Generalizar la fórmula de arriba a más conjuntos no es del todo sencillo y es el  llamado principio del tamiz o principio de inclusión-exclusión (ver apéndice \ref{principiodeltamiz}). 


%\vskip .3cm

\noindent\textbf{El principio de multiplicación}

Suponga que una actividad consiste de $2$ etapas y la primera etapa puede ser realizada de $n_1$ maneras y la etapa $2$  puede realizarse de $n_2$  maneras, independientemente de como se ha hecho la etapa $1$. Entonces toda la actividad puede ser realizada de $n_1\cdot n_2$  formas distintas.


\begin{ejemplo}
Supongamos que la persona del ejemplo \ref{cine} tiene suficiente tiempo y dinero para ir primero al cine y luego al teatro. Entonces tendrá  $3 \cdot 4=12$ formas distintas de hacer el paseo.
\end{ejemplo}


Formalmente, si $A,B$ conjuntos y definimos el {\em producto cartesiano}\index{producto cartesiano} entre $A$ y $B$ por
$$
A \times B = \{(a,b): a \in A, b \in B\}.
$$
Entonces si $A$ y $B$ son conjuntos finitos se cumple que
$$
|A \times B| = |A|\cdot|B|.  
$$

\end{section}

\begin{section}{Selecciones ordenadas con repetición}

Un aplicación inmediata del principio de multiplicación  es que nos permite calcular la cantidad de selecciones ordenadas con repetición. 

\begin{ejemplo} Sea  $X = [[ 1 , 3]] = \{ 1, 2, 3 \}$ ¿de cuántas formas se pueden elegir dos de estos números en forma ordenada?
Es decir, debemos elegir dos números $a$ y $b$ teniendo en cuenta que si $a\not=b$ no es lo mismo elegir $a$ y luego $b$ que $b$ y $a$.  

Para no escribir demasiado vamos a adoptar una notación muy conveniente: si elegimos $a$ y $b$ en forma ordenada, denotamos $ab$. Entonces, en muy breve espacio seremos capaces de escribir todas las selecciones ordenadas de $2$ elementos del  conjunto  $[[ 1 , 3]]$:
\begin{align*}
&11&\quad &12&\quad &13 \\
&21&\quad &22&\quad &23\\
&31&\quad &32&\quad &33
\end{align*}
Son $9 = 3^2$ formas. ¿Cómo justificamos esto? Es claro que para la primera elección tenemos $3$ valores posibles y para la segunda elección tenemos también $3$ valores posibles, entonces, por el principio de multiplicación, tenemos en total $3\cdot 3$ elecciones posibles.  

Avancemos un poco más y ahora elijamos en forma ordenada $3$ elementos de  $[[ 1 , 3]]$, es claro que estas elecciones son
\begin{align*}
&1 1 1&\quad &211&\quad &311 \\
&1 1 2&\quad & 212&\quad & 312\\
&1 1 3&\quad & 213&\quad & 313\\
&1 2 1&\quad & 221&\quad & 321\\
&1 2 2&\quad & 222&\quad & 322\\
&1 2 3&\quad & 223&\quad & 323\\
&1 3 1&\quad & 231&\quad & 331\\
&1 3 2&\quad & 232&\quad & 332\\
&1 3 3&\quad & 233&\quad & 333.
\end{align*}
El total de elecciones posibles $27 = 3^3$. Esto se justifica usando dos veces el principio de multiplicación: para la primera elección tenemos $3$ va\-lo\-res posibles. Para la segunda elección tenemos también $3$ valores posibles, entonces, por el principio de multiplicación, tenemos en total $3\cdot 3$ va\-lo\-res posibles para la elección de los dos primeros números. Como para la tercera elección tenemos $3$ valores posibles, por el principio de multiplicación nuevamente, tenemos   $3\cdot 3 \cdot 3$ elecciones posibles.




Un diagrama arbolado ayuda a pensar.


\begin{tikzpicture}[line width=1pt]	
\lineatz{5}{-95}{28}{-35}
\lineatz{5}{-95}{28}{-95}
\lineatz{5}{-95}{28}{-155}

\ponertz{30}{-35}{1}
\lineatz{32}{-35}{57}{-15}
\lineatz{32}{-35}{57}{-35}
\lineatz{32}{-35}{57}{-50}


\ponertz{60}{-15}{1}
\lineatz{62}{-15}{87}{-10}
\lineatz{62}{-15}{87}{-15}
\lineatz{62}{-15}{87}{-20}
\ponertz{90}{-10}{1}
\ponertz{90}{-15}{2}
\ponertz{90}{-20}{3}
\ponertz{100}{-10}{111}
\ponertz{100}{-15}{112}
\ponertz{100}{-20}{113}

\ponertz{60}{-35}{2}
\lineatz{62}{-35}{87}{-30}
\lineatz{62}{-35}{87}{-35}
\lineatz{62}{-35}{87}{-40}
\ponertz{90}{-30}{1}
\ponertz{90}{-35}{2}
\ponertz{90}{-40}{3}
\ponertz{100}{-30}{121}
\ponertz{100}{-35}{122}
\ponertz{100}{-40}{123}


\ponertz{60}{-55}{3}
\lineatz{62}{-55}{87}{-50}
\lineatz{62}{-55}{87}{-55}
\lineatz{62}{-55}{87}{-60}
\ponertz{90}{-50}{1}
\ponertz{90}{-55}{2}
\ponertz{90}{-60}{3}
\ponertz{100}{-50}{131}
\ponertz{100}{-55}{132}
\ponertz{100}{-60}{133}


\ponertz{30}{-95}{2}
\lineatz{32}{-95}{57}{-75}
\lineatz{32}{-95}{57}{-95}
\lineatz{32}{-95}{57}{-115}

\ponertz{60}{-75}{1}
\lineatz{62}{-75}{87}{-70}
\lineatz{62}{-75}{87}{-75}
\lineatz{62}{-75}{87}{-80}
\ponertz{90}{-70}{1}
\ponertz{90}{-75}{2}
\ponertz{90}{-80}{3}
\ponertz{100}{-70}{211}
\ponertz{100}{-75}{212}
\ponertz{100}{-80}{213}


\ponertz{60}{-95}{2}
\lineatz{62}{-95}{87}{-90}
\lineatz{62}{-95}{87}{-95}
\lineatz{62}{-95}{87}{-100}
\ponertz{90}{-90}{1}
\ponertz{90}{-95}{2}
\ponertz{90}{-100}{3}
\ponertz{100}{-90}{221}
\ponertz{100}{-95}{222}
\ponertz{100}{-100}{223}

\ponertz{60}{-115}{3}
\lineatz{62}{-115}{87}{-110}
\lineatz{62}{-115}{87}{-115}
\lineatz{62}{-115}{87}{-120}
\ponertz{90}{-110}{1}
\ponertz{90}{-115}{2}
\ponertz{90}{-120}{3}
\ponertz{100}{-110}{231}
\ponertz{100}{-115}{232}
\ponertz{100}{-120}{233}


\ponertz{30}{-155}{3}
\lineatz{32}{-155}{57}{-135}
\lineatz{32}{-155}{57}{-155}
\lineatz{32}{-155}{57}{-175}

\ponertz{60}{-135}{1}
\lineatz{62}{-135}{87}{-130}
\lineatz{62}{-135}{87}{-135}
\lineatz{62}{-135}{87}{-140}
\ponertz{90}{-130}{1}
\ponertz{90}{-135}{2}
\ponertz{90}{-140}{3}
\ponertz{100}{-130}{311}
\ponertz{100}{-135}{312}
\ponertz{100}{-140}{313}

\ponertz{60}{-155}{2}
\lineatz{62}{-155}{87}{-150}
\lineatz{62}{-155}{87}{-155}
\lineatz{62}{-155}{87}{-160}
\ponertz{90}{-150}{1}
\ponertz{90}{-155}{2}
\ponertz{90}{-160}{3}
\ponertz{100}{-150}{321}
\ponertz{100}{-155}{322}
\ponertz{100}{-160}{323}

\ponertz{60}{-175}{3}
\lineatz{62}{-175}{87}{-170}
\lineatz{62}{-175}{87}{-175}
\lineatz{62}{-175}{87}{-180}
\ponertz{90}{-170}{1}
\ponertz{90}{-175}{2}
\ponertz{90}{-180}{3}
\ponertz{100}{-170}{331}
\ponertz{100}{-175}{332}
\ponertz{100}{-180}{333}
\end{tikzpicture}


%\vskip .3cm

Cada rama del árbol representa una selección ordenada de elementos de $[[1, 3]]$.

\end{ejemplo}

El razonamiento anterior  se puede extender:

\begin{proposicion}
Sean  $m,n \in \mathbb N$. Hay   $n^m$ formas posibles de elegir ordenadamente $m$ elementos de un conjunto de $n$ elementos.
\end{proposicion}
\begin{proof}[Idea de la prueba]
La prueba de esta proposición se basa en aplicar el principio de multiplicación $m-1$ veces, es decir debemos hacer inducción sobre $m$ y usar el principio de multiplicación en el paso inductivo. 
\end{proof}

\begin{observacion} En el ejemplo denotamos $ [[ 1 , 3]] = \{ 1, 2, 3 \}$. En general, si  $n \in \mathbb N$ denotaremos  $[[ 1 , n]]$ al conjunto de los primeros $n$ números naturales. Es decir:
$$
 [[ 1 , n]] = \{ 1, 2, \ldots,n\}.
$$  
\end{observacion}

\begin{ejemplo}
¿Cuántos números de cuatro dígitos pueden formarse con
los dígitos $1, 2, 3, 4, 5, 6$?

Por la proposición anterior es claro que hay $6^4$ números posibles.
\end{ejemplo}


\begin{ejemplo}
¿Cuántos números de $5$ dígitos y capicúas pueden formarse
con los números dígitos $1, 2, 3, 4, 5, 6, 7, 8$? Un número
capicúa de cinco dígitos es de la forma
$$xyzyx$$
Se reduce a ver cuántos números de tres dígitos pueden
formarse con aquéllos dígitos.
Exactamente $8^3$.
\end{ejemplo}



 

\begin{ejemplo} Sea $X$ un conjunto de $m$ elementos. Queremos contar cuántos subconjuntos tiene este conjunto. 
Por ejemplo, si $X = \{ a, b, c \}$ los subconjuntos de $X$ son exactamente
$$
\emptyset, \{ a \} , \{ b \}, \{ c \}, \{ a, b \}, \{ a, c \}, \{ b, c \}, \{ a, b, c\}.
$$ 
Es decir que existen $8$ subconjuntos de $X$, un conjunto de $3$ elementos. ¿Cómo podemos encontrar razonando este número? Un forma sería la siguiente:  cuando elijo un subconjunto el $a$ puede estar o no estar en el subconjunto, es decir hay dos posibilidades. Con el $b$ pasa lo mismo, puede estar o no estar y por lo tanto hay $2$ posibilidades. Con el $c$ se hace un razonamiento análogo y por lo tanto tenemos que hay en total 
$$
2 \cdot 2 \cdot 2 = 2^3 = 8
$$
posibles subconjuntos de $X$.  

Otra forma de verlo: podemos identificar  cada subconjunto de  $X$ con una terna ordenada de $0$'s y $1$'s de la siguiente manera: si $a$ está en el subconjunto la primera coordenada de la terna es $1$, si no es $0$;  si $b$ está en el subconjunto la segunda coordenada de la terna es $1$, si no es $0$;  si $c$ está en el subconjunto la tercera coordenada de la terna es $1$, si no es $0$. Es decir tenemos la identificación



\begin{align*}
&\emptyset& &\leftrightarrow& &000 \\ 
&\{ a \} & &\leftrightarrow& &100 \\ 
&\{ b \}& &\leftrightarrow& &010 \\ 
&\{ c \}& &\leftrightarrow& &001 \\ 
&\{ a, b \}& &\leftrightarrow& &110 \\ 
&\{ a, c \}& &\leftrightarrow& &101 \\ 
&\{ b, c \}& &\leftrightarrow& &011 \\ 
&\{ a, b, c\}& &\leftrightarrow& &111 .
\end{align*}
 
Observar entonces que seleccionar un subconjunto de $X$ es equivalente a elegir en forma ordenada $3$ elementos del conjunto $\{ 0, 1 \}$; y sabemos entonces que en ese caso tenemos $2^3$ posibilidades. 

En general, cuando $X$ tiene $n$ elementos, 
elegir un subconjunto de $X$ es  equivalente a elegir en forma 
ordenada $n$ elementos del conjunto $\{ 0, 1 \}$ y por lo tanto

\begin{proposicion}\label{cardp} La cantidad de subconjuntos de  
un conjunto de $n$ elementos es $2^n$.
\end{proposicion}

Dado  $X$ un conjunto, denotamos $\mathcal P(X)$ el  conjunto  formado por todos los subconjuntos de $X$, por ejemplo
$$
\mathcal P(\{1,2\}) = \{\emptyset,\{1\},\{2\},\{1,2\}\}.
$$  
Si $X$ es un conjunto finito la proposición \ref{cardp} nos dice que
$$
\mathcal |P(X)| = 2^{|X|}
$$
\end{ejemplo}





\end{section}

\begin{section}{Selecciones ordenadas sin repetición}\label{permutaciones}

Sea $n \in \mathbb{N}$. Recordemos  que  definimos recursivamente \emph{factorial de $n$} al número denotado 
$$n!,$$
tal que
\begin{align*}
1! &= 1\\
(n + 1)! &= n!  (n + 1)
\end{align*}
Definimos también
$$0! = 1$$
Por ejemplo
\begin{align*}
2! &= 2 \cdot 1 = 2 \\
3! &= 3 \cdot 2 \cdot 1 = 6 \\
4! &= 3!\cdot 4 = 6 \cdot 4 = 24. 
\end{align*}


Ahora estudiaremos las selecciones ordenadas de $m$ elementos entre $n$ donde {\em no} se permite la repetición. Es decir si  el conjunto es $A= \{a_1,a_2,\ldots,a_n\}$, las selecciones deben ser del tipo 
$$
a_{i_1} a_{i_2} \cdots a_{i_m}
$$
donde  $a_{i_j} \not= a_{i_k}$ si $i\not=k$. 

Por ejemplo, las selecciones de $3$ elementos en forma ordenada y sin repetición de $[[1, 3]]$  son exactamente
$$
1 2 3,\; 1 3 2,\; 2 1 3,\; 2 3 1,\; 3 1 2,\; 3 2 1
$$
(son las ternas donde los tres números son distintos). O sea hay $6$ selecciones ordenadas y sin repetición de  elementos de $[[1, 3]]$.

Notemos que
$$
3 \cdot 2 \cdot 1 = 6 = 3!
$$
Esta forma de escribir nos da la razón de que haya $6$ selecciones ordenadas y sin repetición de  elementos de $[[1, 3]]$: para la elección del primer elemento tenemos $3$ posibilidades (el $1, 2$ o $3$). Cuando elegimos el segundo elemento, si queremos que no haya repetición, debemos excluir el valor elegido en primer lugar, o sea que tenemos solo $2$ elecciones. Análogamente para la tercera elección solo hay solo una posibilidad, pues debemos descartar los valores elegidos en el primer y segundo lugar. Tenemos entonces  $3 \cdot 2 \cdot 1$ selecciones posibles.

En un diagrama arbolado la selección se puede representar de la siguiente forma:

%\vskip .3cm

\begin{tikzpicture}[line width=1pt]
\lineatz{5}{-35}{27}{-15}
\lineatz{5}{-35}{27}{-35}
\lineatz{5}{-35}{27}{-55}


\ponertz{30}{-15}{1}
\lineatz{32}{-15}{57}{-10}
\lineatz{32}{-15}{57}{-20}
\ponertz{60}{-10}{2}
\ponertz{60}{-20}{3}
\lineatz{62}{-10}{87}{-10}
\lineatz{62}{-20}{87}{-20}
\ponertz{90}{-10}{3}
\ponertz{90}{-20}{2}
\ponertz{100}{-10}{123}
\ponertz{100}{-20}{132}

\ponertz{30}{-35}{2}
\lineatz{32}{-35}{57}{-30}
\lineatz{32}{-35}{57}{-40}
\ponertz{60}{-30}{1}
\ponertz{60}{-40}{3}
\lineatz{62}{-30}{87}{-30}
\lineatz{62}{-40}{87}{-40}
\ponertz{90}{-30}{3}
\ponertz{90}{-40}{1}
\ponertz{100}{-30}{213}
\ponertz{100}{-40}{231}


\ponertz{30}{-55}{3}
\lineatz{32}{-55}{57}{-50}
\lineatz{32}{-55}{57}{-60}
\ponertz{60}{-50}{1}
\ponertz{60}{-60}{2}
\lineatz{62}{-50}{87}{-50}
\lineatz{62}{-60}{87}{-60}
\ponertz{90}{-50}{2}
\ponertz{90}{-60}{1}
\ponertz{100}{-50}{312}
\ponertz{100}{-60}{321}
\end{tikzpicture}


El número total es entonces $3 \cdot 2 \cdot 1 = 6$.

Pensemos ahora que queremos elegir en forma ordenada y sin repetición $3$ elementos entre $5$. Entonces para la primera elección tenemos $5$ posibilidades, para la segunda $4$ posibilidades y para la tercera $3$ posibilidades haciendo un total de 
$$
5 \cdot 4 \cdot 3
$$
selecciones posibles. 


Se puede demostrar que si $n < m$, no hay ninguna selección ordenada y  sin repetición de $m$ elementos  de un conjunto de $n$ elementos (lo cual se ve muy bien intuitivamente: si hay más personas que asientos, ¡alguien se quedará parado!). Este hecho es llamado el {\em principio de las casillas}\index{principio de las casillas} en la literatura.

\begin{proposicion}\label{prop1}
Si $n \ge m$ entonces existen
\begin{equation}\label{ordsinrep}
 n \cdot (n - 1) \cdots (n - (m - 1)), \qquad \text{$m$ - factores}
\end{equation}
selecciones ordenadas y sin repetición de $m$ elementos de un conjunto de $n$ elementos.
\end{proposicion}
\begin{proof} La prueba es una generalización del razonamiento aplicado más arriba a los e\-jem\-plos: debemos seleccionar $m$-veces elementos de un conjunto que tiene $n$ elementos. La primera selección puede ser de cualquiera de los $n$ objetos; la segunda selección debe recaer en uno de los $n-1$ elementos restantes. De manera similar, hay $n-2$ posibilidades para la tercera selección, y así sucesivamente. Cuando hacemos la $m$-ésima selección, $m-1$ elementos ya han sido seleccionados, y entonces el elemento seleccionado debe ser uno de los $n-(m-1)$ elementos restantes. Por consiguiente el número total de
selecciones es el propuesto. 
\end{proof}

\begin{observacion}
El resultado anterior en particular nos dice que existen
$$
n \cdot (n - 1) \cdots (n - (n - 1)) = n \cdot (n - 1) \cdots 1 = n!
$$
selecciones ordenadas y sin repetición de  $n$ elementos en un conjunto con $n$ elementos y esta podría ser una motivación natural del factorial.


Las selecciones ordenadas y sin repetición de  $n$ elementos en un conjunto con $n$ elementos se denominan {\em permutaciones}\index{permutación} de grado $n$.

Hay, pues, $n!$ permutaciones de grado $n$.
\end{observacion}

Volviendo al resultado de la proposición \ref{prop1}, por ejemplo hay 
\begin{itemize}
\item $7 \cdot 6 \cdot 5$ selecciones ordenadas y sin repetición de $3$ elementos de  $[[1,7]]$,
\item $7 \cdot 6 \cdot 5 \cdot 4 \cdot 3$  selecciones ordenadas y sin repetición de $5$ elementos de $[[1,7]]$ y
\item $7!$  selecciones ordenadas y sin repetición de todos los elementos de $[[1,7]]$.
\end{itemize}
%\vskip .3cm

Notemos que si $n \ge m$ entonces
$$
n \cdot (n - 1) \cdots (n - (m - 1)) = \frac{n!}{(n - m)!}
$$
pues
$$
n! = n \cdot (n - 1 ) \cdots (n -(m - 1 ) ) \cdot (n -m)!
$$

Por lo tanto  la proposición \ref{prop1} se puede reescribir de 
la siguiente manera:

% \vskip .5cm

\begin{proposicion}Si $n \ge m$ entonces existen
\begin{equation}\label{ordsinrep2}
\frac{n!}{(n - m)!}
\end{equation}
selecciones ordenadas y sin repetición de $m$ elementos de un conjunto de $n$ elementos.
\end{proposicion}

%\begin{ejercicio} Simplificar las expresiones siguientes ($n \in \mathbb N$)
%\begin{align*}
%&\text{a) } \frac{n!}{( n - 2 ) !} \quad\text{ si } 2 \le n&  & \text{b) } \frac{(n + 2)!}{n!}  \\
%&\text{c) } \frac{(n + 2)!}{( n - 2 ) !} \quad\text{ si } 2 \le n&  & \text{d) } \frac{n!}{(n-2)! 2!}  \quad \text{ si } 2 \le n\\
%&\text{e) } \frac{(n-1)!}{(n + 2)!}& \ &
%\end{align*}
%\end{ejercicio}

\begin{ejemplo}
Si en un colectivo hay $10$ asientos vacíos. ¿De cuántas formas
pueden sentarse $7$ personas? Se trata de ver cuantas selecciones ordenadas y sin repetición de $7$ asientos entre $10$. 

Este número es
$$
10 \cdot 9 \cdot 8 \cdot 7 \cdot 6 \cdot 5 \cdot 4, \qquad \text{$7$ - factores.}
$$
\end{ejemplo}


\begin{ejemplo}
¿Cuántas permutaciones pueden formarse con las letras de
{\em silvia}?

Afirmamos que se pueden formar  $\displaystyle{\frac{6!}{2!}}$ palabras usando las letras de {\em silvia}.

Si escribo en lugar de {\em silvia},
$$
\text{\em s i l v i' a}
$$
Es decir si cambio la segunda {\em i } por {\em i'}, todas las letras son distintas, luego hay $6!$ permutaciones, pero
cada par de permutaciones del tipo
\begin{align*}
\cdots \text{\em i } \cdots  \text{\em i' }  \cdots \\
\cdots \text{\em i' } \cdots  \text{\em i } \cdots
\end{align*}
coinciden, por lo tanto tengo que dividir por $2$ el número total de permutaciones.

Tomemos la palabra
$$
\text{\em ramanathan}
$$
el número total de permutaciones es $\displaystyle{\frac{10!}{ 4!2!}}$.

En efecto, escribiendo el nombre anterior así 
$$
r\;a_1\;m\;a_2\;n_1\;a_3\;t\;h\;a_4\;n_1
$$
el número total de permutaciones es $10!$ Pero
permutando las $a_i$ y las $n_i$ sin mover las otras letras obtenemos
la misma permutación de {\em ramanathan}.

Como hay $4!$ permutaciones de las letras $a_1$, $a_2$, $a_3$, $a_4$, y
$2!$ de $n_1$, $n_2$ el número buscado es 
$$
\frac{10!}{ 4!2!}.
$$

Dejamos a cargo del lector probar que el número total de
permutaciones de las letras de {\em a\-rri\-ve\-der\-ci} es
$$
\frac{11!}{3!  2!  2!}  
$$
\end{ejemplo}



%\begin{subsection}{Ejercicios} 

\subsection*{\Large $\S$ Ejercicios}
%\addcontentsline{toc}{subsection}{Ejercicios}
	Simplificar las siguientes expresiones, sea $n \in \mathbb N$
	\begin{multicols}{2}
\begin{enumerate}[1)]
	\item $\dfrac{n!}{( n - 2 ) !}$, \qquad si $n \geq 2$.
	
	\item $ \dfrac{(n + 2)!}{( n - 2 ) !}$, \qquad si $n \geq 2$.
	
	\item $\dfrac{n!}{(n-2)! 2!} $, \qquad si $n \geq 2$.
	
	\item $\dfrac{(n + 2)!}{n!}$.
	
	\item $\dfrac{(n-1)!}{(n + 2)!}$.
\end{enumerate}			
	\end{multicols}
%\end{subsection}

\end{section}

\begin{section}{Selecciones sin orden}




Consideremos un conjunto $X$ finito de $n$ elementos.
Nos proponemos averiguar cuántos subconjuntos de $m$ elementos
hay en $X$.

\begin{ejemplo}
Por ejemplo, sea $X = \{ 1, 2, 3, 4, 5 \}$ y nos interesan los
subconjuntos de tres ele\-men\-tos. ¿Cuántos habrá? Una forma
de individualizar un subconjunto de tres elementos en $X$, consiste
en, primero, seleccionar  ordenadamente $3$ elementos de $[[ 1 , 5 ]]$.

Habría, a priori, $5 \cdot 4 \cdot 3$ subconjuntos pues ese es el número
de selecciones ordenadas y sin repetición de $3$ elementos de $[[ 1 , 5 ]]$.

Pero es claro que algunas de las selecciones ordenadas pueden determinar el mismo subconjunto. En efecto, por ejemplo, cualesquiera de las selecciones
\begin{align*}
1 2 3, \qquad  1 3 2, \qquad  2 1 3, \qquad 2 3 1, \qquad  3 1 2, \qquad  3 2 1
\end{align*}
determina el subconjunto $\{ 1, 2, 3\}$ . Es decir las permutaciones de $\{ 1, 2, 3\}$ determinan el mismo subconjunto.  Y así con cualquier otro
subconjunto de tres elementos. Por lo tanto, el número total de
subconjuntos de $3$ elementos debe ser
$$
\frac{5 \cdot 4 \cdot 3}{3!} =  \frac{5!}{3! (5 - 3)!}
$$
\end{ejemplo}

%\vskip .3cm

En el caso general de subconjuntos de $m$ elementos de un
conjunto de $n$ elementos ($m \le n$) podemos razonar en forma análoga. Cada
subconjunto de $m$ elementos está determinado por una selección ordenada y todas las permutaciones de esta selección.

Por lo tanto el número total de subconjunto de $m$ elementos
de $X$ es
$$
\frac{n \cdot (n - 1) \cdots (n - (m - 1))}{m!} = \frac{n!}{(n - m)!\; m!}
$$

\begin{definicion}
Sean $n, m \in \mathbb N_0$, $m \le n$. Definimos
$$
\binom{n}{m} = \frac{n!}{(n - m)! \; m!}
$$
y por razones que se verán más adelante se denomina el {\em coeficiente binomial}\index{coeficiente binomial} o {\em número combinatorio}\index{número combinatorio} asociado al par $n$, $m$ con $m \le n$.


Definimos también
$$
\binom{n}{m} = 0,\qquad \text{ si } m > n.
$$
\end{definicion}

\begin{observacion} Hay unos pocos números combinatorios que son fácilmente calculables: 
$$
\binom{n}{0} = \binom{0}{0} = 1 \qquad \text{ y }\qquad  \binom{n}{1} = \binom{n}{n-1} = n. 
$$
Estos resultados se obtienen por aplicación directa de la definición (recordar que  $0! =1$). 
\end{observacion}

En resumen, tenemos la siguiente proposición.

\begin{proposicion}
Sean $n, m \in \mathbb N_0$, $m \le n$, y supongamos que el conjunto $X$ tiene $n$ elementos.
Entonces la cantidad de subconjuntos de $X$ con $m$ elementos es
$
\displaystyle\binom{n}{m}
$.
\end{proposicion}

Como vimos anteriormente el número combinatorio suele resultar de utilidad para resolver pro\-ble\-mas de conteo. Veamos un ejemplo.

\begin{ejemplo}
 ¿Cuántos comités pueden formarse de un conjunto de $6$ mujeres y $4$ hombres, si el comité debe estar compuesto por $4$ mujeres y $2$ hombres?

{\sc Solución.} Debemos elegir $4$ mujeres entre $6$, y la cantidad de elecciones posibles es   $\binom{6}{4}$. Por otro lado, hay $ \binom{4}{2}$ formas de elegir $2$ hombres entre $4$. Luego, por el principio de multiplicación,  el resultado es
$$
\binom{6}{3}\cdot \binom{4}{2} = \frac{6!}{2!4!}\cdot\frac{4!}{2!2!} = \frac{6\cdot 5}{2}\cdot\frac{4\cdot 3}{2} = 15 \cdot 6 = 90.
$$
\end{ejemplo}




\begin{proposicion}[Simetría del número combinatorio]\label{simcomb}
Sean $m,n \in \mathbb N_0$, tal que $m \le n$. Entonces
$$
\binom{n}{m} = \binom{n}{n-m}.
$$
\end{proposicion}
\begin{proof}
$$
\binom{n}{n-m} = \frac{n!}{(n-(n-m))!\,(n-m)!} =  \frac{n!}{m!\,(n-m)!} =   \frac{n!}{(n-m)!\,m!} = \binom{n}{m}.
$$
\end{proof}


\begin{nota}
El hecho  de que 
$$
\binom{n}{m} = \binom{n}{n-m}.
$$
se puede interpretar en términos de subconjuntos:  $\displaystyle\binom{n}{m}$ es el número de subconjuntos de $m$
ele\-men\-tos de un conjunto de $n$ elementos. Puesto que con cada subconjunto de $m$ ele\-men\-tos hay unívocamente asociado un subconjunto de $n - m$ elementos, su complemento en $X$, es claro que $\displaystyle\binom{n}{m} = \binom{n}{n-m}$.
\end{nota}

%\vskip .3cm

\begin{teorema}[Fórmula del triángulo de Pascal] \label{propcomb}
Sean $m,n \in \mathbb N$, tal que $m \le n$. Entonces
$$
\binom{n+1}{m} = \binom{n}{m-1} + \binom{n}{m}  
$$
\end{teorema}
\begin{proof}
El enunciado nos dice que debemos demostrar que 
\begin{equation*}
\frac{(n+1)!}{(n-m+1)!\,m!} = \frac{n!}{(n-m+1)!\,(m-1)!} +  \frac{n!}{(n-m)!\,m!}
\end{equation*}
Hay varias forma de operar algebraicamente las expresiones y obtener el resultado. Nosotros partiremos de la expresión de la derecha y obtendremos la de la izquierda:
\begin{align*}
\frac{n!}{(n-m+1)!\,(m-1)!} +  \frac{n!}{(n-m)!\,m!} &= \frac{n!}{(n-m)!\,(m-1)!}\left(\frac{1}{(n-m+1)} +  \frac{1}{m}\right)\\
&= \frac{n!}{(n-m)!\,(m-1)!}\left(\frac{m+n-m+1}{(n-m+1)m} \right) \\
& = \frac{n!}{(n-m)!\,(m-1)!}\left(\frac{n+1}{(n-m+1)m} \right) \\
& = \frac{n!(n+1)}{(n-m)!(n-m+1)\,(m-1)!\,m}\\
& = \frac{(n+1)!}{(n-m+1)!\,m!}.
\end{align*}

\end{proof}


Aunque por razones de conteo es obvio que los números combinatorios son números naturales, esto no es claro por la definición formal.  
\begin{corolario}
Si $n \in \mathbb N$ y $ 0\le m \le n$ entonces $\displaystyle\binom{n}{m} \in \mathbb N$.
\end{corolario}
\begin{proof}
Haremos inducción en $n$. Si $n = 1$ los posibles números
combinatorios son
$$
\binom{1}{1}  = \binom{1}{0}  = 1 \in \mathbb N.
$$

Ahora supongamos que el resultado sea cierto  para $n \in \mathbb N$. Es decir, $\binom{n}{m} \in \mathbb N$ cualquiera sea $m$ tal que $0 \le m \le n$ (hipótesis inductiva). Probaremos entonces que 
\begin{equation*}
	\binom{n+1}{m} \in \mathbb N
\end{equation*}
 para todo  $m \in \mathbb{N}$ tal que $0 \le m \le n+1$. 
 
Consideremos tres casos: $m=0$,\; $1 \le m \le n$\; y \;$m = n+1$.

Si $m=0$, entonces
\begin{equation*}
	\binom{n+1}{m} = \binom{n+1}{0} =1 \in \mathbb N.
\end{equation*}


Si $1 \le m \le n$, entonces por el teorema anterior (en la segunda parte)
$$
\binom{n+1}{m} = \binom{n}{m-1} + \binom{n}{m}  
$$
Como 
$\binom{n}{m-1}$  y $\binom{n}{m}$ pertenecen  a $\mathbb  N$ por la hipótesis inductiva, su suma es también un número natural, o sea 
\begin{equation*}
	\binom{n+1}{m}   \in \mathbb N
\end{equation*}
para $1 \le m \le n$.


Si $m = n+1$, entonces 
\begin{equation*}
	\binom{n+1}{n+1} = 1  \in \mathbb N.
\end{equation*}


Por lo anterior, se concluye que
$$
\binom{n+1}{m}   \in \mathbb N
$$
cualquiera sea $m$ , $0 \le m \le n + 1$.

Por lo tanto, es válido el paso inductivo y así nuestra afirmación
queda probada.
\end{proof}

El teorema precedente permite calcular los coeficientes
binomiales inductivamente. Escribamos en forma de triángulo


\begin{align*}
&& && && && && &\binom{0}{0}& && && && && &&  \\
&& && && && &\binom{1}{0}& && &\binom{1}{1}& && && && &&  \\
&& && && &\binom{2}{0}& && &\binom{2}{1}& && &\binom{2}{2}& && && &&  \\
&& && &\binom{3}{0}& && &\binom{3}{1}& && &\binom{3}{2}& && &\binom{3}{3}& && &&  \\
&& &\binom{4}{0}& && &\binom{4}{1}& && &\binom{4}{2}& && &\binom{4}{3}& && &\binom{4}{4}& &&  \\
&\cdot& && &\cdot& && &\cdot& && &\cdot& && &\cdot& && &\cdot& 
\end{align*}



En virtud del teorema \ref{propcomb} cada término interior es
suma de los dos términos inmediatos superiores. Los elementos
en los lados valen 1 por lo tanto se puede calcular cualquiera
de ellos.


\begin{align*}
&& && && && && &1& && && && && &&  \\
&& && && && &1& && &1& && && && &&  \\
&& && && &1& && &2& && &1& && && &&  \\
&& && &1& && &3& && &3& && &1& && &&  \\
&& &1& && &4& && &6& && &4& && &1& &&  \\
&\cdot& && &\cdot& && &\cdot& && &\cdot& && &\cdot& && &\cdot& 
\end{align*}
(Lector: calcule el valor de la suma total de cada fila del triángulo.)

El  triángulo es denominado \emph{triángulo de Pascal}. Entre las propiedades que cuenta el triángulo de Pascal está la de ser simétrico respecto de su altura, como consecuencia de la simetría de los números combinatorios (ver proposición \ref{simcomb}). 





\end{section}


\begin{section}{El teorema del binomio}

En álgebra elemental aprendemos las formulas
$$
(a+b)^2 = a^2 +2ab +b^2, \qquad (a+b)^3 = a^3 + 3 a^2b +3ab^2 +
b^3,
$$
y a veces nos piden desarrollar la formula para $(a+b)^4$ y
potencias mayores de $a+b$. El resultado general que da una
formula para $(a+b)^n$ es conocido como el
 {\it {teorema del binomio}}.  \index{Teorema del binomio}

\begin{teorema}\label{t3.6}
Sea $n$ un entero positivo. El coeficiente del termino
$a^{n-r}b^r$ en el desarrollo de $(a+b)^n$ es el número binomial
$\binom{n}{r}$. Explícitamente,
\begin{equation*}
(a+b)^n= \binom{n}{0} a^n + \binom{n}{1} a^{n-1}b+ \binom{n}{2}
a^{n-2}b^2 + \cdots + \binom{n}{n} b^n,
\end{equation*}
o,  escrito en forma más concisa, 
\begin{equation}\label{eq-th-bin-1}
(a+b)^n= \sum_{i=0}^{n}\binom{n}{i} a^{n-i}b^i.
\end{equation}

\end{teorema}
\begin{proof}(Primera) Considerar que ocurre cuando
multiplicamos $n$ factores
$$
(a+b)(a+b) \cdots (a+b).
$$
Un término en el producto se obtiene seleccionando o bien $a$ o
bien $ b$ de cada factor. El número de términos $a^{n-r}b^r$ es
solo el número de formas de seleccionar $r$ $b$'s (y
consecuentemente $n-r$ a's), y por definición éste es el número
binomial $\binom{n}{r}$.
\end{proof}


\begin{observacion}\label{cvar} Antes de hacer una segunda demostración del teorema del binomio veamos el siguiente resultado que nos resultará útil: sea $a_k,a_{k+1},\ldots,a_{m-1},a_m$ una sucesión de números reales ($k \le m$) y sea $r \in \mathbb N_0$.  Entonces
$$
\sum_{i=k}^m a_i = \sum_{i=r}^{m-k+r} a_{i+k-r}.
$$ 
La sumatoria de la derecha es la de la izquierda con un ``cambio de variable'' en el índice. La demostración de este hecho se puede hacer por inducción sobre $m$ (caso base $m=k$) o simplemente escribiendo ambas sumatorias con la notación de puntos suspensivos y verificando que ambas son iguales a
$$
a_k+a_{k+1}+\cdots+a_{m-1}+a_m.
$$  
\end{observacion}


\begin{observacion}
	Por la simetría del número binomial la  fórmula (\ref{eq-th-bin-1}) es equivalente a 
	\begin{equation*}
	(a+b)^n= \sum_{i=0}^{n}\binom{n}{n-i} a^{n-i}b^i.
	\end{equation*}
	Haciendo  el cambio de variable $i \leftrightarrow n-i$, se obtiene la fórmula equivalente
	\begin{equation*}\label{eq-th-bin-2}
	(a+b)^n= \sum_{i=0}^{n}\binom{n}{i} a^ib^{n-i}.
	\end{equation*}
	La fórmula anterior es también, muy utilizada para enunciar el teorema del binomio.
\end{observacion}



\begin{proof}[Demostración (*)](Segunda) Se hace por inducción en $n$. Si
$n=1$, el resultado es trivial. Supongamos que el resultado es
cierto para $n-1$, es decir,  de acuerdo  a la fórmula (\ref{eq-th-bin-1}),
$$
(a+b)^{n-1}=\sum_{i=0}^{n-1} \binom{n-1}{i}a ^{n-1-i}b^{i}.
$$
Luego
\begin{alignat*}2
(a+b)^n&= (a+b)(a+b)^{n-1}&& \\
& = (a+b)\left(\sum_{i=0}^{n-1} \binom{n-1}{i}a ^{n-1-i}b^{i}\right)&&\qquad \text{por hip. inductiva} \\
&=\sum_{i=0}^{n-1} \binom{n-1}{i}a ^{n-i}b^{i}+\sum_{i=0}^{n-1} \binom{n-1}{i}a ^{n-1-i}b^{i+1}&& \qquad \text{propiedad distributiva}\\
&=\sum_{i=0}^{n-1} \binom{n-1}{i}a ^{n-i}b^{i}+\sum_{i=1}^{n} \binom{n-1}{i-1}a ^{n-i}b^{i}&& \qquad \text{ver observación \ref{cvar}}\\
&= a^n + \sum_{i=1}^{n-1}\left\{ \binom{n-1}{i}+\binom{n-1}{i-1}\right\}a ^{n-i}b^{i}+ b^n && \qquad \text{agrupar por potencias iguales}\\
&= a^n + \sum_{i=1}^{n-1} \binom{n}{i}a ^{n-i}b^{i}+ b^n&&\qquad \text{por teorema 3.4.1} \\
&= \sum_{i=0}^{n} \binom{n}{i}a ^{n-i}b^{i}.&&\text{}
\end{alignat*}
\end{proof}

%

Los coeficientes en el desarrollo pueden ser calculados con el método recursivo usado para los números binomiales (triángulo de Pascal) o usando la formula. Por ejemplo,
\begin{equation*}
	\begin{aligned} (a+b)^6 &= \binom{6}{0} a^6 + \binom{6}{1} a^5 b
	+\binom{6}{2}a^4b^2 +
	\binom{6}{3}a^3b^3 \\
	&\quad + \binom{6}{4}a^2b^4+\binom{6}{5}ab^5\binom{6}{6}b^6 \\
	&=  a^6 + 6 a^5 b +15a^4b^2 + 20a^3b^3 + 15a^2b^4+6ab^5+ b^6.
	\end{aligned}
\end{equation*}


Por supuesto, podemos obtener otras formulas útiles si
reemplazamos $a$ y $b$ por otras expresiones. Algunos ejemplos
típicos son:
$$\begin{aligned}
(1+x)^4 &= 1 + 4x + 6x^2+ 4x^3+ x^4;\\
(1-x)^7 &= 1 -7x + 21x^2- 35x^3+ 35x^4- 21x^5+ 7x^6 -x^7;\\
(x+2y)^5 &= x^5 + 10 x^4 y + 40 x^3 y^2+80 x^2 y^3+80 x y^4+32 y^5; \\
(x^2+y)^4 &= x^8 +4 x^3 y +6 x^4 y^2 +4 x^2 y^3 + y^4.
\end{aligned}
$$




La expresión $a+b$ es conocida como un expresión {\it binómica}
porque tiene dos términos. Como los números  \index{expresión
binómica} $\binom{n}{r}$ aparecen como los coeficientes en el
desarrollo de $(a+b)^n$, generalmente se los llama, como ya fue dicho, 
coeficientes binomiales.  
De todos modos esta claro por la primera prueba del teorema \ref{t3.6} que
estos números aparecen en este contexto porque representan el
número de formas de hacer ciertas selecciones. Por esta razón
continuaremos usando el nombre de números binomiales,
 que se aproxima más al concepto que
simbolizan.

Además de ser extremadamente útil en manipulaciones algebraicas,
el teorema del binomio puede usarse para deducir identidades en
que estén involucrados los números binomiales.

\begin{ejemplo} Probemos que 
	\begin{equation*}
	\sum_{i=0}^{n}\binom{n}{i} = \binom{n}{0}+\binom{n}{1}+\binom{n}{3}+\cdots+\binom{n}{n}= 2^n.
	\end{equation*}
\end{ejemplo}
\begin{proof}
	Observar  que $1 + 1 = 2$, luego, por el teorema del binomio,
	\begin{align*}
		2^n = (1 +1)^n 
		    &= \sum_{i=0}^{n} \binom{n}{i}1^{n-i}1^i \\
		    &= \sum_{i=0}^{n} \binom{n}{i}1 \cdot 1 \\
		    &= \sum_{i=0}^{n} \binom{n}{i}.
	\end{align*} 
\end{proof}

\begin{ejemplo}Probemos que
\begin{equation}\label{eqcuaddo}
\binom{n}{0}^2+\binom{n}{1}^2+\binom{n}{3}^2+\cdots+\binom{n}{n}^2=
\binom{2n}{n}.
\end{equation}
\end{ejemplo}
\begin{proof}
Usamos la igualdad
\begin{equation*}\label{eqxn}
(1+x)^n(1+x)^n=(1+x)^{2n}.
\end{equation*}
El resultado se demostrará encontrando el coeficiente del termino $x^n$ de ambos términos de esta igualdad.

De acuerdo con el teorema del binomio el miembro izquierdo es el
producto de dos factores, ambos iguales a
$$
\binom{n}{0}1+\binom{n}{1}x+\cdots+\binom{n}{r}x^r+\cdots+\binom{n}{n}x^n.
$$
 Cuando los dos factores se multiplican, un termino en $x^n$ se
obtiene tomando un termino del primer factor de tipo $x^i$ y un termino del
segundo factor de tipo $x^{n-i}$. Por lo tanto los coeficientes de $x^n$ en el
producto son
\begin{equation}\label{igcuad}
\binom{n}{0}\binom{n}{n}+\binom{n}{1}\binom{n}{n-1}+\binom{n}{2}\binom{n}{n-2}+\cdots
+\binom{n}{n}\binom{n}{0}.
\end{equation}
Como $\binom{n}{n-r}=\binom{n}{r}$, vemos que (\ref{igcuad}) es el lado
izquierdo de la igualdad (\ref{eqcuaddo}). Pero el lado derecho es
$\binom{2n}{n}$ que es también el coeficiente de $x^n$ en el
desarrollo de $(1+x)^{2n}$, y entonces obtenemos la igualdad que
buscábamos.
\end{proof}

\vskip .3cm 

\begin{observacion}[Definición de $0^0$] No hay consenso matemático en cual es el valor de $0^0$, pues dependiendo del contexto puede ser conveniente definir su valor de cierta forma o declarar su valor como \textit{indeterminado} (ver el artículo de Wikipedia  \href{https://es.wikipedia.org/wiki/Cero_a_la_cero}{Cero  a la cero}). 
	
	Sin embargo,  para su uso  en álgebra, así como en combinatoria se considera conveniente  definir  $0^0 =1$. ¿Cuál es la razón para darle este valor? Existen varias razones, pero una de las más importantes proviene del teorema del binomio: observemos que 
	\begin{equation*}
		(1 + x)^n = \sum_{i=0}^{n}\binom{n}{i} x^i
	\end{equation*}
	En particular, para $x=0$, la ecuación es
	\begin{equation*}
	1^n = \sum_{i=0}^{n}\binom{n}{i} 0^i.
	\end{equation*}
	Es claro que para $i>0$, $0^i = 0 \cdot \ldots \cdot 0 =0$, luego la ecuación se reduce a
	\begin{equation*}
	1 = 1^n =  0^0.
	\end{equation*}
	No sería muy convincente enunciar el teorema del binomio con algunas excepciones poco naturales (como ser, exigir que $a$ y $b$ sean no nulos) y la formulación más general parece ser la correcta. Esta formulación exige que $0^0 =1$, como hemos visto más arriba.
	
	Una fundamentación extensa y rigurosa sobre la conveniencia de definir $0^0 =1$ se encuentra en el artículo de D. Knuth \textit{Two notes on notation} que se encuentra en \href{https://arxiv.org/abs/math/9205211}{https://arxiv.org/abs/math/9205211}.
\end{observacion}

%\begin{subsection}{Ejercicios}
\subsection*{\Large $\S$ Ejercicios}\label{ej3.6.1}
%\addcontentsline{toc}{subsection}{Ejercicios}	

	
	
\begin{enumerate}[1)]
\item Desarrollar las fórmulas de $(1+x)^8$ y $(1-x)^8$.

\item Calcular los coeficientes de:
\begin{multicols}{2}
\begin{enumerate}[a)]
	\item $x^5$ en $(1+x)^{11}$;
	
	\item $a^2b^8 $ en $(a+b)^{10}$;
	
	\item $a^6 b^6$ en $(a^2 +b^3)^5$;
	
	\item $x^3$ en $(3+4x)^6$.
\end{enumerate}	
\end{multicols}
\item Usar la identidad $(1+x)^m(1+x)^n=(1+x)^{m+n}$ para probar que
$$
\binom{m+n}{r} =
\binom{m}{0}\binom{n}{r}+\binom{m}{1}\binom{n}{r-1}+\cdots
\binom{m}{r}\binom{n}{0}
$$
donde $m,n$ y $r$ son enteros positivos y, $m\ge r$, y $n \ge r$.
%\end{enumerate}
%\end{subsection}
%\end{section}
%\begin{section}{Ejercicios}
\end{enumerate}


\section{Ejercicios}

\begin{enumerate}[1)]

\item Una ficha del juego de dominó puede ser representada con el símbolo $[x|y]$,
donde $x$ e $y$ son miembros del conjunto $\{0,1,2,3,4,5,6\}$. Los números $x$
e $y$ pueden ser iguales. Explicar por que el número total de fichas de dominó
es $28$ y no $49$.

\item ?`De cuántas formas se pueden elegir un casillero negro y uno blanco en
un tablero de ajedrez de tal forma que los dos casilleros no estén ni en la misma
fila ni en la misma columna?

\item Suponer que en clase hay $m$ mujeres y $n$ varones. ?`De cuántas formas
puedo ordenar a todos los alumnos en una hilera de manera que todas las mujeres
queden juntas?

\item Suponer que tenemos un dominó generalizado en que las fichas toman su
par de valores entre $0$ y $n$. Sea $k$ un entero en el rango $0\le k \le n$.
Probar que el número de fichas $[x|y]$ en que $x+y = n-k$ es igual al número de
fichas en que  $x+y = n+k$.

\item Denotar $u_n$ la cantidad de palabras de longitud $n$ en el alfabeto
$\{0,1\}$ que tienen la propiedad de no tener dos ceros consecutivos. Probar que
$$
u_1=2,\qquad u_2=3,\qquad u_n =u_{n-1} +u_{n-2},\qquad n \ge 3.
$$

\item Desarrollar $(x+y)^9$ y $(x-y)^9$.

\item Calcular el coeficiente de:
\begin{multicols}{3}
\begin{enumerate}[a)]
	\item $x^6$ en $(1+x)^{12}$;
	
	\item $a^3b^7$ en $(a+b)^{10}$;
	
	\item $a^4b^6$ en $(a^2+b)^8$
\end{enumerate}
\end{multicols}
%$$
%\begin{aligned}
%\text{(i)}\quad & x^6 \text{ en } (1+x)^{12},\\
%\text{(ii)}\quad & a^3 b^7 \text{ en } (a+b)^{10},\\
%\text{(iii)}\quad & a^4 b^6 \text{ en } (a^2+b)^8.
%\end{aligned}
%$$

\item Probar que
$$
\binom{n}{r}\binom{r}{k}=\binom{n}{k}\binom{n-k}{r-k}.
$$

\item Se dibujan todas las posibles diagonales que conectan a un conjunto de $n$
puntos en el círculo y se observa que no hay tres rectas que se cruzan en un
mismo punto ?`Cuántos puntos internos de intersección hay?

\item Probar que el número de formas de distribuir $n$ bolas idénticas en $m$
cajas con etiquetas, algunas de las cuales puede quedar vacía es
$$
\binom{n+m-1}{n}.
$$

\item Probar que si $n \ge m$, entonces
$$
\binom{m}{m}+\binom{m+1}{m}+\cdots+\binom{n}{m}=\binom{n+1}{m+1}.
$$

\item  Sea $X$ un $n$-conjunto. Probar que

\begin{enumerate}[a)]
	\item existe un conjunto de $\binom{n-1}{k-1}$ $k$-subconjuntos de
$X$ tales que cada par de ellos tiene intersección no vacía.

	\item existe un conjunto de $\binom{n}{n^*}$ subconjuntos de $X$ 
con la propiedad que ninguno contiene a otro. Aquí $n^*$ es igual
a $n/2$ si $n$ es par y a $\frac12(n-1)$ si $n$ es impar.
\end{enumerate}
\end{enumerate}
%\end{subsection}

\end{section}
