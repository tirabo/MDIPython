\documentclass[12pt,spanish,makeidx]{amsbook}
\tolerance=10000
\renewcommand{\baselinestretch}{1.3}



\usepackage{t1enc}
\usepackage[spanish]{babel}
\usepackage{latexsym}
\usepackage[utf8]{inputenc}
\usepackage{verbatim}
\usepackage{multicol}
\usepackage{amsgen,amsmath,amstext,amsbsy,amsopn,amsfonts,amssymb}
\usepackage{calc}         % From LaTeX distribution
\usepackage{graphicx}     % From LaTeX distribution
\usepackage{ifthen}       % From LaTeX distribution
\usepackage{subfigure}    % From CTAN/macros/latex/contrib/supported/subfigure
\usepackage{pst-all}      % From PSTricks
\usepackage{pst-poly}     % From pstricks/contrib/pst-poly
\usepackage{multido}      % From PSTricks
\usepackage{fancyhdr}
\input{random.tex}        % From CTAN/macros/generic


%% \theoremstyle{plain} %% This is the default
\oddsidemargin 0.0in \evensidemargin -1.0cm \topmargin 0in
\headheight .3in \headsep .2in \footskip 0.5in
\setlength{\textwidth}{16cm} %ancho para apunte
\setlength{\textheight}{21cm} %largo para apunte
%\leftmargin 2.5cm
%\rightmargin 2.5cm
\topmargin 0.5 cm

\pagestyle{fancy}
\fancyhf{}
\fancyhead[LE,RO]{FAMAF}
\fancyhead[RE,LO]{Matemática Discreta I}
\fancyfoot[LE,RO]{\leftmark}
\fancyfoot[RE,LO]{\thepage}
 
\renewcommand{\headrulewidth}{0.5pt}
%\renewcommand{\footrulewidth}{0.5pt}
 



\begin{document}



{\bf \begin{center} Práctico 1 \\ Matemática Discreta I -- Año 2020/1 \\ FAMAF \end{center}}

\smallskip

\begin{enumerate}


\item\label{prob1} Demostrar las siguientes afirmaciones donde $a$, $b$, $c$ y $d$ son siempre números enteros. Justificar cada uno de los pasos en cada demostración indicando el axioma o resultado que utiliza.
\begin{enumerate}
\item  $a=-(-a)$
\item  $a=b\,$ si y sólo si $\,-a=-b$
\item  $a+a=a$ implica que  $a=0$.
\end{enumerate}


\medskip

\item Idem \ref{prob1}.

\begin{enumerate}
 \item $0<a\,$ y $\,0<b\,$ implican $\,0<a\cdot b$
 \item $a<b\,$ y $\,c<0$ implican $\,b\cdot c<a\cdot c$
\end{enumerate}

\medskip

\item  Probar las siguientes afirmaciones, justificando los pasos que realiza.
\begin{enumerate}
  \item Si $0 < a$  y $\,0<b\,$ entonces $\,a<b\,$ si y sólo si $a^2<b^2$.
  \item Si $a\neq 0$  entonces $a^2>0$.
  \item Si $a\neq b$  entonces $a^2+b^2>0$.
  \item Probar que si $a+c <b+c$ entonces $a<b$.
\end{enumerate}

\medskip


\item Sea $u_1=3$, $u_2=5$ y $u_n=3 u_{n-1} - 2 u_{n-2}$ con $n\in \mathbb N$, $n\geq 3$.
Probar que $u_n=2^n+1$.

\smallskip

\item Sea $\{ u_n \}_{n \in \mathbb N}$ la sucesión definida por recurrencia como sigue: $u_1 = 9$, $u_2 = 33$, $u_n = 7u_{n-1} - 10u_{n-2}$, $\forall n \geq 3$. Probar que $u_n = 2^{n+1} + 5^n$, para todo $n \in \mathbb N$.



\smallskip

\item Probar que $\sum_{i=0}^n 2^i = 2^{n+1} -1$ ($n \ge 0$). 


\smallskip

\item Sea $u_n$ definida recursivamente por: $u_1=2$, $u_n=2+\sum_{i=1}^{n-1}2^{n-2i}u_i \;\;\forall\; n >1$.
\begin{enumerate}
 \item Calcule $u_2$ y $u_3$.
 \item Proponga una fórmula para el término general $u_n$ y pruébela por inducción.
\end{enumerate}


\smallskip

\item Probar las siguientes afirmaciones usando inducción en $n$:
\begin{enumerate}
\item $n^2\leq 2^n$ para todo $n\in{\mathbb N}$, $n>3$ .
%\item $n^3 \le 3^n$;\quad $\forall n \in {\mathbb N}$, $n\ge 3$ .
%\item $(n+1)^n < n^{n+1}$; $\forall n \in {\mathbb N}$, $n \ge 3$.
\item $\forall n \in {\mathbb N}$,\ $3^n \ge 1 + 2^n$.
\end{enumerate}

\smallskip


\item Calcular evaluando las siguientes expresiones:
\begin{multicols}{4}
 \begin{enumerate}
\item \quad $\displaystyle{\sum_{r=0}^4 r}$
\item \quad $\displaystyle{\prod_{i=1}^5 i}$
\item  \quad $\displaystyle{\sum_{k=-3}^{-1} \frac{1}{k(k+4)}}$
\item \quad $\displaystyle{\prod_{n=2}^7 \frac{n}{n-1}}$
\end{enumerate}
\end{multicols}




\smallskip

\item Calcular:
\begin{multicols}{2}
 \begin{enumerate}
\item \quad $2^{10} - 2^{9}$
\item \quad $3^2 2^5 - 3^5 2^2$
\item \quad $(2^2)^n - (2^n)^2$
\item \quad $(2^{2^n} + 1)  (2^{2^n} - 1)$
\end{enumerate}
\end{multicols}




\smallskip




\item Dado un natural $m$, probar que $\forall n \in {\mathbb N} $; $x$, $y \in {\mathbb R}$, se cumple:
\begin{multicols}{3}
 \begin{enumerate}
  \item $x^n \cdot x^m = x^{n+m}$
\item $(x\cdot y)^n=x^n\cdot y^n$
\item $(x^n)^m = x^{n\cdot m}$
 \end{enumerate}
\end{multicols}



\smallskip

\item Analizar la validez de las siguientes afirmaciones:
%\begin{multicols}{2}
 \begin{enumerate}
\item  $(2^{2^n})^{2^k} = 2^{2^{n+k}}$,  $n$, $k \in {\mathbb N}$.
\item $(2^n)^2 = 4^n$, $n \in {\mathbb N}$.
\item $2^{7+11} = 2^7 + 2^{11}$.
\end{enumerate}
%\end{multicols}

\smallskip


\item Demostrar por inducción  las siguientes igualdades:
  \begin{enumerate}
  \item  $\displaystyle{ \sum_{k=1}^n (a_k + b_k) = \sum_{k=1}^n a_k + \sum_{k=1}^n b_k}$, $n\in \mathbb N$.
  \item  $\displaystyle{ \sum_{j=1}^n j = \frac{n(n+1)}{2}}$, $n\in \mathbb N$, $n\in \mathbb N$.
  \item  $\displaystyle{ \sum_{i=1}^n i^2 = \frac{n(n+1)(2n+1)}{6}}$, $n\in \mathbb N$.
  \item  $\displaystyle{ \sum_{k=0}^n (2k+1) = (n+1)^2}$, $n\in \mathbb N_0$.
  \item  $\displaystyle{ \sum_{i=1}^n i^3 = \left( \frac{n(n+1)}{2 }\right)^2}$, $n\in \mathbb N$.
  \item  $\displaystyle{ \sum_{k=0}^n a^k = \frac{a^{n+1}-1}{a-1}}$, donde $a\in {\mathbb R}$, $a \neq 0,\ 1$, $n\in \mathbb N_0$.
  \item  $\displaystyle{ \prod_{i=1}^n \frac{i+1}{i} = n+1}$, $n\in \mathbb N$.

  \item $\displaystyle{ \sum_{i=1}^n \frac{1}{4i^2-1} = \frac{n}{2n+1}}$, $n\in \mathbb N$.

  \item $\displaystyle{ \sum_{i=1}^n i^2\, /\, \sum_{j=1}^n j = \frac{2n+1}{3}}$, $n\in \mathbb N$.

  \item $\displaystyle{ \prod_{i=2}^n \left(1-\frac{1}{i^2}\right) = \frac{n+1}{2n}}$, $n\in \mathbb N$ y $ n\ge 2$.

  \item Si $a\in \mathbb R$ y $a\geq -1$, entonces $(1+a)^n\geq 1+n\cdot a$, $\forall \, n \in \mathbb N$.

  \item Si $a_1,\dots,a_n \in \mathbb R$, entonces $\displaystyle{\sum_{k=1}^n a_{k}^{2}\leq \left(\sum_{k=1}^n |a_{k}|\right)^{2}}$, $n\in \mathbb N$.


  \item Si $a_1,\dots,a_n \in \mathbb R$ y $0<a_i<1 \forall \, i$, entonces $(1-a_1)\cdots(1-a_n)\ge 1-a_1-\cdots -a_n$, $n\in \mathbb N$.

  \end{enumerate}

\smallskip




\item Hallar $n_0 \in {\mathbb N}$ tal que $\forall n \ge n_0$ se cumpla que $n^2 \ge 11 \cdot n + 3$.

\smallskip


\item Las siguientes proposiciones no son válidas para todo $n \in {\mathbb N}$. Indicar en qué paso del principio de inducción falla la demostración:
\begin{multicols}{4}
\begin{enumerate}
\item  $n=n^2$,
\item  $n=n+1$,
\item  $3^n = 3^{n+2}$,
\item  $3^{3n} = 3^{n+2}$.
\end{enumerate}
\end{multicols}

\smallskip

\item Encuentre el error en los siguientes argumentos de inducción.
\begin{enumerate}
\item  Demostraremos que $5n+3$ es múltiplo de 5 para todo $n\in \mathbb N$.

Supongamos que $5k+3$ es múltiplo de 5, siendo $k\in \mathbb N$. Entonces existe
$p\in \mathbb N$ tal que  $5k+3=5p$. Probemos que $5(k+1)+3$ es múltiplo de 5:
Como
$$
5(k+1)+3=(5k+5)+3=(5k+3)+5=5p+5=5(p+1),
$$
entonces obtenemos que $5(k+1)+3$ es múltiplo de 5. Por lo tanto, por el principio
de inducción, demostramos que $5n+3$ es múltiplo de 5 para todo $n\in \mathbb
N$.

\smallskip

\item Sea $a\in\mathbb R$, con $a\neq 0$. Vamos a demostrar que para todo entero no negativo $n$, $a^n=1$.

Como $a^0=1$ por definición, la proposición es verdadera para $n=0$. Supongamos
que para  un entero $k$, $a^m=1$ para $0\leq m \leq k$. Entonces
$a^{k+1}= \frac{a^k a^k}{a^{k-1}}=\frac{1\cdot1}1=1$.
Por lo tanto, el principio de inducción fuerte implica que $a^n=1$ para todo $n\in \mathbb N$.
\end{enumerate}


\end{enumerate}


\end{document}
