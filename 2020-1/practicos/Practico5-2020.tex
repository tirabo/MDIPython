\documentclass[12pt,spanish,makeidx]{amsbook}
\tolerance=10000
\renewcommand{\baselinestretch}{1.3}



\usepackage{t1enc}
\usepackage[spanish]{babel}
\usepackage{latexsym}
\usepackage[utf8]{inputenc}
\usepackage{verbatim}
\usepackage{multicol}
\usepackage{amsgen,amsmath,amstext,amsbsy,amsopn,amsfonts,amssymb}
\usepackage{calc}         % From LaTeX distribution
\usepackage{graphicx}     % From LaTeX distribution
\usepackage{ifthen}       % From LaTeX distribution
\usepackage{subfigure}    % From CTAN/macros/latex/contrib/supported/subfigure
\usepackage{tikz}
\usepackage{tkz-graph}
\usepackage{tikz-3dplot} %for tikz-3dplot functionality
\usepackage{pgfplots}
\usetikzlibrary{graphs}
\usetikzlibrary{matrix}
\input{random.tex}        % From CTAN/macros/generic


%% \theoremstyle{plain} %% This is the default
\oddsidemargin 0.0in \evensidemargin -1.0cm \topmargin 0in
\headheight .3in \headsep .2in \footskip .2in
\setlength{\textwidth}{16cm} %ancho para apunte
\setlength{\textheight}{21cm} %largo para apunte
%\leftmargin 2.5cm
%\rightmargin 2.5cm
\topmargin 0.5 cm



\usepackage{hyperref}
\hypersetup{
    colorlinks=true,
    linkcolor=blue,
    filecolor=magenta,      
    urlcolor=cyan,
}
\usepackage{hypcap}

\newcommand\itemitem[1]{\item[#1]}


\newcommand{\ExampleSubFigure}[2][0.3333]{%
\subfigure[Example #2]{%
  \begin{minipage}[t]{#1\textwidth}
    \parbox[b]{\textwidth}{%
      \centering
      \input{lgc/#2.inl}}
  \end{minipage}}}


%% \theoremstyle{plain} %% This is the default
\oddsidemargin 0.0in \evensidemargin -1.0cm \topmargin 0in
\headheight .3in \headsep .2in \footskip .2in
\setlength{\textwidth}{16cm} %ancho para apunte
\setlength{\textheight}{21cm} %largo para apunte
%\leftmargin 2.5cm
%\rightmargin 2.5cm
\topmargin 0.5 cm



%% \theoremstyle{plain} %% This is the default



\renewcommand{\thesection}{\thechapter.\arabic{section}}
\renewcommand{\thesubsection}{\thesection.\arabic{subsection}}




\newtheorem{teorema}{Teorema}[section]
\newtheorem{proposicion}[teorema]{Proposici\'on}
\newtheorem{corolario}[teorema]{Corolario}
\newtheorem{lema}[teorema]{Lema}

\theoremstyle{definition}

\newtheorem{definicion}{Definici\'on}[section]
\newtheorem{ejemplo}{Ejemplo}[section]
\newtheorem{problema}{Problema}[section]
\newtheorem{ejercicio}{Ejercicio}[section]
\newtheorem{ejerciciof}{}[section]

\theoremstyle{remark}
\newtheorem{observacion}{Observaci\'on}[section]
\newtheorem{nota}{Nota}[section]



\renewcommand{\abstractname}{Resumen}
\renewcommand{\partname }{Parte }
\renewcommand{\indexname}{Indice }
\renewcommand{\figurename }{Figura }
\renewcommand{\tablename }{Tabla }
\renewcommand{\proofname}{Demostraci\'on}
\renewcommand{\refname }{Referencias }
\renewcommand{\appendixname }{Ap\'endice }
\renewcommand{\contentsname }{Contenidos }
\renewcommand{\chaptername }{Cap\'\i tulo }
\renewcommand{\bibname }{Bibliograf\'\i a }





\def\conc{+\hspace{-1.5ex}+\hspace{0.5ex}}
\def\con{{\rm con}}
\def\hn{\hspace{-0.2cm}}
\def\h4n{\hspace{-0.4cm}}
\def\impli{\Rightarrow}
\def\ssi{\equiv}
\def\disc{\not\ssi}
\def\cons{\Leftarrow}
\def\la{\leftarrow}
\def\lt{\triangleleft}
\def\lv{[\;\,]}
\def\Max{ {\rm Max} }
\def\Min{ {\rm Min} }
\def\N{I\hspace{-0.8ex} N}
\def\noi{\noindent}
\def\R{I\hspace{-0.8ex} R}
\def\ra{\rightarrow}
\def\Rn{\R^{n}}
\def\rt{\triangleright}
%\def\tomar{\uparrow}
\def\tomar{\hspace{-0.6ex}\uparrow\hspace{-0.6ex}}
%\def\tirar{\downarrow}
\def\tirar{\hspace{-0.6ex}\downarrow\hspace{-0.6ex}}
\def\udo{ {\rm\bf\underline{do}} }
\def\ufi{ {\rm\bf\underline{fi}} }
\def\uif{ {\rm\bf\underline{if}} }
\def\uod{ {\rm\bf\underline{od}} }
\def\v3{\vspace{0.3cm}}
\def\var{{\rm var}}
\def\[{|\hspace{-0.2ex} [}
\def\]{]\hspace{-0.2ex} |}
\def\true{\mbox{\it true\ }}
\def\false{\mbox{\it false\ }}

\newcommand{\bag}[1]{ [\hspace{-0.4ex}[ #1 ]\hspace{-0.4ex}] }
\newcommand{\abs}[1]{ [\hspace{-0.4ex}[ #1 ]\hspace{-0.4ex}] }
%\newcommand{\binom}[2]{ \left(\hspace{-1.2ex}\begin{array}{c} #1 \\ #2 \end{array} \hspace{-1.2ex}\right) }
\newcommand{\cau}[2]{\noi #1 \hspace{0.5cm}\{{\sl #2}\} }
\newcommand{\causa}[2]{\vspace{0.15cm} \noi #1 \hspace{0.5cm} \{{\sl #2}\} \vspace{0.15cm}}
\newcommand \RR{{\mathbb R}}
\newcommand \ZZ{{\mathbb Z}}
\newcommand \NN{{\mathbb N}}
%\newcommand \supr{\displaystyle{\ \,{\mbox{\footnotesize S}}\hspace{-1.7ex}\bigcirc\, }}
\newcommand \supr{\displaystyle{\ \,\vee \hspace{-1.9ex}\bigcirc\,}}
\newcommand \infi{\displaystyle{\ \,\wedge\hspace{-1.9ex}\bigcirc\,}}
%\newcommand \infi{\displaystyle{\ \,{\mbox{\footnotesize I}}\hspace{-1.5ex}\bigcirc\, }}
\newcommand \mcd{\operatorname{mcd}}
\newcommand \mcm{\operatorname{mcm}}
\newcommand \sisolosi{\Leftrightarrow}






\newcommand\flecha{\Rightarrow}

\newcommand{\programa}[1]{  \vskip .5cm \noindent Programa: {\it #1} }
\newcommand{\principio}[1]{\vskip .3 cm \centerline{\sc #1} \vskip .3cm}



\usepackage{fancyhdr}
\pagestyle{fancy}
\fancyhf{}
\fancyhead[LE,RO]{FAMAF}
\fancyhead[RE,LO]{Matemática Discreta I}
\fancyfoot[LE,RO]{\leftmark}
\fancyfoot[RE,LO]{\thepage}
 
\renewcommand{\headrulewidth}{0.5pt}
%\renewcommand{\footrulewidth}{0.5pt}
 



\begin{document}

{\bf \begin{center} Práctico 5 \\ Matemática Discreta I -- Año 2020/1 \\ FAMAF \end{center}}

\medskip




\begin{enumerate}

\item  ¿Cuántas aristas tiene el  grafo completo $K_n$? ¿Para cuáles valores de $n$ se puede encontrar un dibujo de $K_n$ con la propiedad que las líneas representan las aristas sin cruzarse?


\medskip
\item Encuentre un isomorfismo entre los grafos por las siguientes listas. (Ambas listas especifican versiones de un famoso grafo conocida como {\it grafo de Petersen}.)
$$
\begin{matrix}
a&b&c&d&e&f&g&h&i&j\\ \hline
b&a&b&c&d&a&b&c&d&e\\
e&c&d&e&a&h&i&j&f&g\\
f&g&h&i&j&i&j&f&g&h
\end{matrix}
\qquad \begin{matrix}
0&1&2&3&4&5&6&7&8&9\\ \hline
1&2&3&4&5&0&1&0&2&6\\
5&0&1&2&3&4&4&3&5&7\\
7&6&8&7&6&8&9&9&9&8
\end{matrix}
$$

\medskip
\item
\begin{enumerate}
\item Encuentre todos los grafos de 5 vértices y 2 aristas no isomorfos entre sí.
 \item ¿Cuál es el máximo número de aristas que puede tener un grafo de 5 vértices?
\end{enumerate}

\medskip

\item Para cada una de las siguientes secuencias, encuentre un grafo que tenga exactamente las valencias indicadas o demuestre que tal grafo no existe:
\begin{multicols}{3}
\begin{enumerate}
\item $3,3,1,1$
\item $3,2,2,1$
\item $3,3,2,2,1,1$
\item $4,1,1,1,1$
\item $7,3,3,3,2,2$
\item $4,1,1,1$
\end{enumerate}
\end{multicols}

\medskip

\item Demuestre que los siguientes pares de grafos son isomorfos (encuentre un isomorfismo):





\begin{tabular}{ll}
	${}^{}$ \qquad &
\begin{tikzpicture}[scale=1]
\draw (-1,2) node {(a)};
\SetVertexSimple[Shape=circle, FillColor=white,MinSize=8 pt]
\SetVertexNoLabel
\Vertex[]{A}
\Vertex[x=1.5,y=0]{B}
\Vertex[x=3,y=0]{C}
\Vertex[x=1.5,y=1.5]{D}
\Vertex[x=1.5,y=-1.5]{E}
%
\Edges(A,D,C,E,A)
\Edges(A,B,C)
\Edges(D,B)

\Vertex[x=4.5,y=0.5]{2}
\Vertex[x=6,y=0.5]{3}
\Vertex[x=7.5,y=0.5]{4}
\Vertex[x=4.5,y=-1]{5}
\Vertex[x=6,y=-1]{6}
\Edge[style={bend left}](2)(4)
\Edges(2,3,4,6,5,2)
\Edges(4,3,6)
\end{tikzpicture}
\end{tabular}



\begin{tabular}{ll}
	${}^{}$ \qquad &
	\begin{tikzpicture}[scale=1]
	\draw (-1,1) node {(b)};
	\SetVertexSimple[Shape=circle, FillColor=white,MinSize=8 pt]
	%\SetVertexNoLabel
	\Vertex[x=0,y=0]{A}
	\Vertex[x=1.5,y=0.8]{B}
	\Vertex[x=3,y=0]{C}
	\Vertex[x=1.5,y=-0.8]{D}
	\Vertex[x=0,y=-0.8]{E}
	\Vertex[x=1.5,y=0]{F}
	\Vertex[x=3,y=-0.8]{G}
	\Vertex[x=1.5,y=-1.6]{H}
	%
	\Edges(A,B,C,D,A)
	\Edges(E,F,G,H,E)
	\Edges(A,E)
	\Edges(B,F)
	\Edges(C,G)
	\Edges(D,H)

	
	\Vertex[x=4.5,y=0]{1}
	\Vertex[x=5.5,y=0]{2}
	\Vertex[x=6.5,y=0]{3}
	\Vertex[x=7.5,y=0]{4}
	\Vertex[x=4.5,y=-1]{5}
	\Vertex[x=5.5,y=-1]{6}
	\Vertex[x=6.5,y=-1]{7}
	\Vertex[x=7.5,y=-1]{8}
	\Edge[style={bend left}](1)(4)
	\Edges(1,2,3,4,8,7,6,5,1)
	\Edges(2,6,7,3)
	\Edge[style={bend right}](5)(8)
	\end{tikzpicture}
\end{tabular}




\medskip

\item Sean $G=(V,E)$ y $G^{\prime}=(V^{\prime},E^{\prime})$ dos grafos y sea $\alpha :V \mapsto V^{\prime}$ una función tal que $\delta (v)=\delta (\alpha (v)) \;\;\forall\,\; v \in V$.
\begin{enumerate}
	\item ¿Puede afirmar que $\alpha $ es un isomorfismo?.
	\item ¿Puede afirmarlo si $|V|=3$ ó 4?.
\end{enumerate}



\item Encuentre una función del grafo $A$ al $B$ que preserve valencias. ¿Es un isomorfismo?.


\begin{tabular}{llll}
	$A:$ & &\qquad$B:$& \\
	 &
	\begin{tikzpicture}[scale=1]
	\SetVertexSimple[Shape=circle,FillColor=white,MinSize=8 pt]
	\Vertex[x=0,y=0]{A}
	\Vertex[x=3,y=0]{B}
	\Vertex[x=3,y=-3]{C}
	\Vertex[x=0,y=-3]{D}
	\Vertex[x=1,y=-1]{E}
	\Vertex[x=2,y=-1]{F}
	\Vertex[x=2,y=-2]{G}
	\Vertex[x=1,y=-2]{H}
	\Edges(A,B,C,D,A)
	\Edges(E,F)
	\Edges(G,H)
	\Edges(A,E,G,C)
	\Edges(B,F,H,D)
	\end{tikzpicture}
	&
	& \begin{tikzpicture}[scale=1]
	\SetVertexSimple[Shape=circle,FillColor=white,MinSize=8 pt]
	\Vertex[x=0,y=0]{A}
	\Vertex[x=3,y=0]{B}
	\Vertex[x=3,y=-3]{C}
	\Vertex[x=0,y=-3]{D}
	\Vertex[x=1,y=-1]{E}
	\Vertex[x=2,y=-1]{F}
	\Vertex[x=2,y=-2]{G}
	\Vertex[x=1,y=-2]{H}
	\Edges(A,B,C,D,A)
	\Edges(E,F,G,H,E)
	\Edges(A,E)
	\Edges(B,F)
	\Edges(C,G)
	\Edges(H,D)
	\end{tikzpicture}
\end{tabular}



\item Pruebe que si $G$ es un grafo con más de un vértice, entonces existen dos vértices con la misma valencia.



\item Si $G=(V,E)$ grafo,  el \textit{grafo complemento}  es $G' = (V,E')$, donde $E'$ son todos los 2-subconjuntos de $V$ que no están en $E$. Es decir, el grafo complemento tiene los mismos vértices que el grafo original y todas las aristas que le faltan a $G$ para ser grafo completo. 
\begin{enumerate}
	\item  Halle el complemento de los siguientes grafos:
	


	\begin{tikzpicture}[scale=1]
	
	\SetVertexSimple[Shape=circle,FillColor=white,MinSize=8 pt]
	
	\draw (-0.8,0) node {(a)};
	\Vertex[x=0,y=0]{A}
	\Vertex[x=0,y=-1]{B}
	\Edges(A,B)
	
	
    \draw (1.2,0) node {(b)};
	\Vertex[x=2,y=0]{A}
	\Vertex[x=3,y=-0.5]{B}
	\Vertex[x=2,y=-1]{C}
	\Edges(A,B,C,A)
	
	\draw (4.2,0) node {(b)};
	\Vertex[x=5,y=0]{A}
	\Vertex[x=6.5,y=0]{B}
	\Vertex[x=6.5,y=-1.5]{C}
	\Vertex[x=5,y=-1.5]{D}
	\Edges(A,B,C,D,A,C)
	
	\draw (7.2,0) node {(b)};
	\Vertex[x=8,y=-1]{A}
	\Vertex[x=9.5,y=0]{B}
	\Vertex[x=11,y=0]{C}
	\Vertex[x=12.5,y=-1]{D}
	\Vertex[x=11,y=-2]{E}
	\Vertex[x=9.5,y=-2]{F}
	\Edges(A,B,C,D,E,F,A)
	\Edges(B,C,E,F,B)
	\Edges(A,C)

	\end{tikzpicture}

	

\medskip

	\item  Si $V=\{ v_1 \dots v_n \} $ y $\delta (v_i)=d_i \;\;\forall\,\; i=1,\dots , n\,$, calcule las valencias de el grafo complemento.
\end{enumerate}

\medskip




\item Pruebe que los siguientes grafos no son isomorfos:


\begin{tabular}{llll}
	&
	\begin{tikzpicture}[scale=1]
	\SetVertexSimple[Shape=circle,FillColor=white,MinSize=8 pt]
	\Vertex[x=0.00, y=2.00]{a}
	\Vertex[x=2., y=-1.50]{b}
	\Vertex[x=-2., y=-1.50]{c}
	\Edges(a,b,c,a)
	\Vertex[x=0.00, y=0.85]{1}
	\Vertex[x=1., y=-0.9]{2}
	\Vertex[x=-1., y=-0.9]{3}
	\Edges(1,2,3,1)
	\Edges(a,1,3,c,b,2)
	\end{tikzpicture}
	&
	\qquad
	& 
	\begin{tikzpicture}[scale=0.65]
	\SetVertexSimple[Shape=circle,FillColor=white,MinSize=8 pt]
	%
	\Vertex[x=3.00, y=0.00]{1}
	\Vertex[x=1.50, y=2.60]{2}
	\Vertex[x=-1.50, y=2.60]{3}
	\Vertex[x=-3.00, y=0.00]{4}
	\Vertex[x=-1.50, y=-2.60]{5}
	\Vertex[x=1.50, y=-2.60]{6}
	\Edges(1,2,3,4,5,6,1)
	\Edges(1,4) \Edges(3,6) \Edges(2,5)
	\end{tikzpicture}
\end{tabular}



\medskip

\item

\item Dados los siguientes grafos:

\begin{tikzpicture}[scale=0.8]
%\SetVertexSimple[Shape=circle,FillColor=white,MinSize=8 pt]

\draw (-0.8,0.2) node {(1)};
\Vertex[x=0,y=-1.5, L=$a$]{A} %\draw (-0.50,-1.5) node {$a$};
\Vertex[x=1.5,y=0, L=$b$]{B}  %\draw (1,0) node {$b$};
\Vertex[x=3,y=-1.5, L=$c$]{C}
\Vertex[x=1.5,y=-3, L=$d$]{D}
\Vertex[x=1.5,y=-1.5, L=$e$]{E}
\Edges(A,B,C,D,A)
\Edges(A,E,C)
\Edges(B,E,D)

\draw (4.2,0.2) node {(2)};
\Vertex[x=5,y=-1.5]{1}
\Vertex[x=6.5,y=0]{2}
\Vertex[x=8.5,y=-1.5]{3}
\Vertex[x=5.7,y=-2.5]{4}
\Vertex[x=7.7,y=-2.7]{5}
\Edges(1,2,3,4,1,5)
\Edges(1,3)
\Edges(2,4)

\draw (10,0.2) node {(3)};
\Vertex[x=12.8,y=0]{1}
\Vertex[x=11.8,y=-1.2]{2}
\Vertex[x=3+10.8,y=-1.2]{3}
\Vertex[x=0+10.8,y=-2.4]{4}
\Vertex[x=4+10.8,y=-2.4]{5}
\Vertex[x=1+10.8,y=-3.6]{6}
\Vertex[x=3+10.8,y=-3.6]{7}
\Vertex[x=2+10.8,y=-3.8]{8}
\Edges(1,6,8,7,1)
\Edges(3,2,7,8,6,4,5)
\end{tikzpicture} 

\begin{tikzpicture}[scale=0.8]
\draw (-0.8,-5.5) node {(4)};
\Vertex[x=0+0.2,y=0-5.9]{1}
\Vertex[x=3+0.2,y=0-5.9]{2}
\Vertex[x=3+0.2,y=-3-5.9]{3}
\Vertex[x=1.5+0.2,y=-3-5.9]{4}
\Vertex[x=0+0.2,y=-3-5.9]{5}
\Edges(1,2,3,4,5,1)
\Edges(2,4,1,3)

\draw (4.2,-5.3) node {(5)};
\Vertex[x=0+5,y=-1.5-5.9, L=$a$]{A}
\Vertex[x=1.5+5,y=0-5.9, L=$b$]{B}
\Vertex[x=3+5,y=-1.5-5.9, L=$c$]{C}
\Vertex[x=1.5+5,y=-3-5.9, L=$d$]{D}
\Vertex[x=1.5+5,y=-1.5-5.9, L=$e$]{E}
\Edges(A,B,C,D,A)
\Edges(A,E,C)


\draw (10,-5.3) node {(6)};
\Vertex[x=0+5+5.8,y=-1.5-5.9, L=$a$]{A}
\Vertex[x=1.5+5+5.8,y=0-5.9, L=$b$]{B}
\Vertex[x=3+5+5.8,y=-1.5-5.9, L=$c$]{C}
\Vertex[x=1.5+5+5.8,y=-3-5.9, L=$d$]{D}
\Vertex[x=1.5+5+5.8,y=-1.5-5.9, L=$e$]{E}
\Edges(A,B,C,E,A,D,E)
\end{tikzpicture} 

\begin{tikzpicture}[scale=0.7]
\draw (-0.8,-10) node {(7)};
\Vertex[x=2,y=-11]{1}
\Vertex[x=4,y=-11]{2}
\Vertex[x=6,y=-11]{3}
\Vertex[x=0,y=-13]{4}
\Vertex[x=5,y=-14]{5}
\Vertex[x=6.3,y=-13]{6}
\Vertex[x=2,y=-15]{7}
\Vertex[x=8,y=-15]{8}
\Vertex[x=3,y=-16]{9}
\Vertex[x=7,y=-16]{10}
\Vertex[x=3,y=-18]{11}
\Vertex[x=5,y=-18]{12}
\Edges(1,6,12,9,1)
\Edges(4,2,5)
\Edges(3,7,11,10,8,3)
\end{tikzpicture} 



\begin{enumerate}
\item Determine en cada caso si existen subgrafos completos de más de 2 vértices.
\item Para el grafo (1), dé todos los caminos que unen $a$ con $b$.
\item Dé caminatas eulerianas en los grafos (4), (5) y (6).
\item Para (2) y (3), decir si existen ciclos hamiltonianos.
\item Determinar cuales de los siguientes pares de grafos son isomorfos:

 (i) (4) y (2),\quad

 (ii) (5) y (6), \quad

 (iii) (5) y (1).



\item Halle las componentes conexas del grafo (7).
\end{enumerate} 

\medskip

\item Dado el siguiente grafo
$$
\begin{matrix}
0&1&2&3&4&5&6&7&8\\ \hline
1&0&1&0&3&0&1&0&1\\
3&2&3&2&5&4&5&2&3\\
5&6&7&4&&6&7&6&5\\
7&8&&8&&8&&8&7.
\end{matrix}
$$
encuentre un ciclo hamiltoniano (si existe). Determine si existe una caminata euleriana y en caso de ser así encuentre una. 

\medskip

\item Un ratón intenta comer un $3\times 3\times 3$ cubo de queso. Él comienza en una esquina y come un subcubo de $1\times 1\times 1$, para luego pasar a un subcubo adyacente. ¿Podrá el ratón terminar de comer el queso en el centro?


\item Dé todos los árboles de 5 vértices no isomorfos.



\end{enumerate}
\end{document}